\chapter{Vocabulary for web application domain}

\section{Elements of the vocabulary}


\subsection{Data definitions}

To understand the necessity of certain elements of the vocabulary, further details of web the applications have to be explained. In VIVO, the display of the existing and the creation of the new data happens in individual pages. The display is done by so-called profile pages, that show the information about one particular instance. As it was in section \ref{233}, the information is grouped by predicates, and each predicate field contains a link that can call the data input pages. The link contains three parameters, \textit{subjectUri}, \textit{predicateUri} and the \textit{rangeUri}.The \textit{subjectUri} hold the value of the instance on whose profile the link is, the \textit{predicateUri} is for the predicate, with which the new dataset is connected to the subject, and the \textit{rangeUri} is an optional paramater.

\img{\IV}{1_1.png}{Complete workflow of data input process}{411_1}


Figure \ref{411_1} shows the workflow of a data entry process. On the right profile page of a skeletal inventory can be seen, which lists the added skeletal elements. The profile page is configured that the \textit{rangeUri} parameter holds in the links the URI of the classes of  skull and vertebral column respectively. These parameters have to be considered by the form loading because they influences the options of the first selector which let the user add the skeletal sub sub division. So therefore it necessary to introduce a flag into the vocabulary that sign if a variable is coming with the HTTP request for the form loading, or with the JSON object after the submission.

\img{\IV}{1_2.png}{Variable types and their attributes}{411_1}


\img{\IV}{1_3.png}{Variable types and their attributes}{411_1}

\subsection{Form definition}


\img{\IV}{1_4.png}{Classes for form}{411_1}



\section{Use-cases of the \textit{RDFBones} project}

\subsection{Skeletal Inventories}

\subsection{Study Design Execution}
