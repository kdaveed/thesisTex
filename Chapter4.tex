\chapter{Implementation}


\section{Web application ontology}


\subsection{Introduction}

\begin{itemize}
	\item Computer programs are able to overtake exercise from humans to accelerate work. In this case work we want speed up is the application development. 
\end{itemize}

\img{\IVI}{basicFlow.png}{Basic workflow}

\begin{itemize}
	\item  For example we have the form definition.
\end{itemize}


\begin{lstlisting}[language=html, basicstyle=\footnotesize]
<form action="http://example.org/newUser">
   	Username <input type="text" name="userName"></br>
	Password <input type="text" name="password"></br>
	<input type="submit">Submit</input>
</form>
\end{lstlisting}

\itemI{Let the developer concentrate on the meaningful parts.}

\begin{lstlisting}[basicstyle=\footnotesize]
{
  type : "form",
  action : "http://example.org/newUser",
  elements : [
    { type : "text", text : "Username", varName : "userName"},
    { type : "text", text : "Password", varName : "password"}
  ]
}
\end{lstlisting}

\img{\IVI}{frameworkFunc.png}{Framework functionality}



\itemI{The goal is store the configuration of the web application persistently in a database. In our case the goal is to store the data in RDF data.}

\subsection{Form definition}

\img{\IVI}{ontologyForm.png}{Web application ontology for the form}

\itemI{Explanation of the main elements and their functionality}

\itemII{Next that each form element has to represent a variable.  This is denoted with blue the wa:Variable. And the variable name from Image X. is taken by variable instance, which is represented by the form}{Here is important to explain the difference between the wa:SubFormAdder and its two subclasses, wa:ClassSelectorSubformAdder, and wa:InstanceSelectorSubformAdder. }

\img{\IVI}{difference.png}{Difference between sub form adders}

\itemI{min, max cardinality will be here addressed}

\img{\IVI}{formDescRDF.png}{Form descriptor RDF Data}





\subsection{Data definition}

\img{\IVI}{statement.png}{Statement in RDF Vocabulary}

\img{\IVI}{coreOntology.png}{Core ontology}

\img{\IVI}{variableTypes.png}{Variable types}

\img{\IVI}{ontologyStatement.png}{Statement types and attributes}

\img{\IVI}{tripleSet1.png}{Statement configuration dataset I}

\img{\IVI}{tripleSet2.png}{Statement configuration dataset II.}

\img{\IVI}{formDefinition1.png}{Statement definition I}

\subsection{VIVO adoption}

\itemI{As it was described in the previous section in VIVO each custom entry form is called by an HTTP request that contains three parameters: subjectUri, predicateUri and objectUri.}

\img{\IVI}{base.png}{Base definition of the data input process}

\itemI{So the key point is that the data input process instances can be queried by the property coming with the request.}

\section{Server-side implementation}



