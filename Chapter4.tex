\chapter{Vocabulary for web application domain}


\section{Elements of the vocabulary}

\subsection{Data definitions}

To understand the necessity of certain elements of the vocabulary, further details of web the applications have to be explained. In VIVO, the display of the existing and the creation of the new data happens in individual pages. The display is done by so-called profile pages, that show the information about one particular instance. As it was in section \ref{233}, the information is grouped by predicates, and each predicate field contains a link that can call the data input pages. The link contains three parameters, \textit{subjectUri}, \textit{predicateUri} and the \textit{rangeUri}.The \textit{subjectUri} hold the value of the instance on whose profile the link is, the \textit{predicateUri} is for the predicate, with which the new dataset is connected to the subject, and the \textit{rangeUri} is an optional paramater.

\img{\IV}{1_1.png}{Complete workflow of data input process}{411_1}


Figure \ref{411_1} shows the workflow of a data entry process. On the right profile page of a skeletal inventory can be seen, which lists the added skeletal elements. The profile page is configured that the \textit{rangeUri} parameter holds in the links the URI of the classes of  skull and vertebral column respectively. These parameters have to be considered by the form loading because they influences the options of the first selector which let the user add the skeletal sub sub division. So therefore it necessary to introduce a flag into the vocabulary that sign if a variable is coming with the HTTP request for the form loading, or with the JSON object after the submission.

Figure \ref{41_3} shows all the nodes types and their attributes. Above the \textit{mainInput} boolean flag, there is the variable name, which is required, and the constant value in case. There are three types of variable, the class, resource and literal. The literal variable itself denotes a string value and it has several subclasses for the other primitive types.

\img{\IV}{1_2.png}{Variable types and their attributes}{41_2}

The other, more important is the modeling of the triples in the dataset. The vocabulary for triples has the purpose to express different constraint on the data scheme as well. Above the class \textit{Triple} and \textit{MultiTriple}, which were addressed in the end of the last chapter, there are two types of restriction triples. One for the classes and one for the instances. As we have seen in the description of the OWL ontologies, there are different types of restrictions can be defined. For this reason it should be possible to allow the definition of which restriction is used by the ontology, upon the entry form should operate. This can be expressed by the three boolean, types of the \textit{classRestrictionTriple}. Moreover the greedy boolean flag means that the SPARQL query that queries the ontology has to return not only the result class but their superclasses too. Finally the instance restriction triples as the name indicates, expresses constraints between instances on the form. The examples in the next section will make the usage of the vocabulary more clear.

\img{\IV}{1_3.png}{Triple types}{41_3}

\subsection{Form definition}

The class \textit{Form} acts like a container for the form elements. There are two main types of form elements, the literal field and the selector. The literal field do not have further subclasseses because its type, is defined through the type of the variable it represents. It would be a sort of over definition. This is the simples case where the form adopts to the data model.

The selector can refer both instances and classes. If it represent and instance, so it let the selection of an existing instance, then it is possible to define an \textit{InstanceViewer}. This feature allows to define a table with several columns. The utility is that the application can show more information about an instance than only the label in the selector field. Each column has a title and a number, and they refer to as well \textit{RDFNodes}, whose values they show in the entries.

The class \textit{SubformAdder} is the subclass of the selector, and has a relationship to the form with the predicate sub form. With this connection it is possible to define the sub form as new form instance. Moreover it has boolean flag, that allow to define a button add all should appear on, which adds all the possible subforms. This feature is useful be the skeletal subdivisions that contains much bones.


\img{\IV}{1_4.png}{Form definition}{411_1}


\section{Use-cases of the \textit{RDFBones} project}

The aim of this section is to show how it is possible to solve different problems with designed vocabulary. It covers the two main tasks of the project, the skeletal inventories and the study design execution. These two examples are sufficient to show the utility of the particular elements in the vocabulary and exemplify how their assembly can lead to a compact definition of complex web application problems.

\subsection{Skeletal Inventories}

Skeletal inventories were already addressed in section \ref{[331]}. Their goal is to define what kind of skeletal elements are present. In this part the creation of the primary skeletal inventories are discussed in more detail. This use-case addresses some additional challenges of the software, that were not mentioned yet.
The explanation starts with the illustration of the implemented interface, to show what problem the high level logic has to define. As it was mentioned in figure \ref{411_1}, the entry forms can have inputs, which in this case the input is the class URI of the skeletal subdivision that has to be added with it sub subdivisions and bone organs. Therefore the first element of the entry form is the selector of the sub subdivision.


\img{\IV}{SI_UI.png}{Input form for skeletal inventories}{SI_UI}

Next to the selector the buttons \textit{Add} and \textit{Add all} can be seen, that let the user add the sub forms. \figref{SI_UI} shows the layout, when two sub sudivision were added, to each of them, a bone organ. The bone organ selector works exactly the same as the sub sudvision selector, but the selector for the completeness state is simple selector, without a sub form. 

\figref{SI_UI} shows the triple scheme representing the skeletal inventories, where the nodes are representing variables. All the rectangle are representing instances. All of them will be newly created instead of the subjectUri, which comes as a main input, and the arrows with double line depicts the multi triples. Important to note that between the variable \textit{boneOrgan} and \textit{boneSegment} there is only a single triple (\textit{fma:regional\_part\_of}) because in this use-case is simplified version and only entire bone segments will be added. 

\img{\IV}{SI_DATA.png}{Skeletal inventory data triples}{SI_DATA}



Above the instances to be created the classes have to be represented in the model too, because they can be the subjects of the class restriction triples.
 Figure \figref{SI_DATA_C} is depicts the complete data model of the problem.

\img{\IV}{SI_DATA_C.png}{Complete data definition}{SI_DATA_C}

For the better readability the predicates are not denoted, but their value can be found in figure \figref{SI_DATA}. Each instance (rectangles) is connected to the class variable (ellipses). The red dotted arrows indicates restriction statements. The large three rectangles, that encompass set of triples are the graph, which are connecting to each other by means of multi triples. This is the structure which is followed by the form as well.

\img{\IV}{SI_FORM.png}{Form layout definition}{SI_FORM}



Above the auxililiary rectangles for the graph,the nodes are colored to indicate their role in the process. The information that the colors hold is not defined in the vocabulary but it can be inferred from the whole data and form model. The first rule is that the main input nodes (\textit{subjectUri} and \textit{rangeUri}) are denoted with red, while the variables appear on the interface are light red. Base on that information is it already possible to determine to which instances it is required to assign an unused URI. Those instance are denoted with yellow.
Furthermore there are two classes in the data model that do not appear on the interface, but their values can be evaluated through SPARQL queries. These are denoted with blue. And the classes with without color, do not appear on the final dataset, but they indicates constraint on the existing instances.
Finally \figref{SI_FORM} display the configuration data describing the form structure. 


\subsection{Study Design Execution}


The entry form for study design execution has as well the hierarchical layout like the one for skeletal inventories, but there are additionally two elements on the main form. The first is the selector from skeletal inventories. It plays a role by the selection of the bone organs as input for the assays. To each assay a set bone segment types is defined in the extensions, that can be can be assigned to them as input. These bone on this form are not created newly but existing ones are selected, that were already added in the frame of the skeletal inventory data input.  However there can be a large amount bone segments stored in the system, and thus the search is facilitated by showing only the ones that belong to the preselected skeletal inventory. The second is a global label field, whose value will be the label of all newly created instances.




\img{\IV}{SDE_UI.png}{Input form for study design execution}{SDE_UI}

Moreover the bone segment selector is not just a HTML selector input field, but a floating window implemented by JavaScript that allows the convenient browsing (\figref{SDE_SI}). It has two advantage with respect to the conventional selector. Firstly it allows to display additional information about the instances above their labels, like their types or longer descriptions. Secondly it does not loads the form layout with additional subform for the selected instances, which by large amount assays and measurement datums is an important aspect.

\img{\IV}{SDE_IS.png}{Instance selector for existing bone segment}{SDE_SI}


On \figref{SDE_SI} can be seen that the there are two section, one for the selected instances, and one for the instances to select. 


The complete data structure of the form can be seen on the following image.

\img{\IV}{SDE_DATA_C.png}{Complete data model}{SDE_DATA_C}


