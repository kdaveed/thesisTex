

\chapter{Conclusion}

\section{Evaluation}

The applied VIVO framework offered possibility to define data input processes in a declarative way to some limited extent by defining the elements of the form and RDF data pattern. The simplicity lied in that VIVO allowed the definition of literals of particular instances. This required only static HTML forms and simple variable substitution algorithms. However during the \textit{RDFBones} project the emphasis was not on the literals but on the instances, and there were such cases where multiple instance had to be created through one data entry form, so that their relationship was one one-to-many. This required dynamic interfaces with some handler algorithm on them instead of static HTML forms, and the server algorithm had to become more complex. Important challenge moreover that rules regarding which entity belongs to which was declared in ontology extension, thus interface had generate its elements dynamically based on ontological data.

During thesis an extended vocabulary were designed that able to express the more complex problems and the code library for the client and the server too which were able to manage the functionality. By the data definition the utility of the thesis that the designed vocabulary is able to express if a the subject and the object of a particular triple in the graph pattern of the data input process are in one-to-one or in one-to-many relationship with each other (\textit{Triple} vs. \textit{MultiTriple}). Additionally this scheme was possible the reflected in the input forms definition which allowed the definition of sub form (\textit{SubformAdder}). This allowed the user to create multi dimensional dataset on the forms and which could be processed by the server based on the extended graph definitions. Further advancement that vocabulary for data definition did not only relate to the RDF triples that were supposed to be stored, but to RDF triples as well which contained described the system, namely the OWL restriction in the ontology extensions (\textit{RestrictionTriple}). As the interface elements are the representations of the RDF nodes of the graph pattern, the framework did know how they are dependent on them, and the server and client routines together could solve this issues.

With the achieved system it is made possible to allow the user to define such data input processes, where the elements of the interface adopts to the ontology extensions, without having to care about how the client and server routines solve that data flow between the user and database. This approach can significantly accelerate the development and allow the Semantic Web based application development for individuals without experience with web application and programming technologies.





\section{Future work}


