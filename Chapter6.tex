

\chapter{Conclusion}

\section{Evaluation}

The applied VIVO framework offered the possibility to define data input processes in a declarative way to some limited extent, by defining the elements of the form and RDF data pattern. The simplicity lied in that VIVO allowed the input of literals of particular instances, which required only static HTML forms and simple value substitution algorithms. However during the \textit{RDFBones} project the emphasis was not on the literals but on the instances, and there were such cases where multiple instance had to be created through one data entry form, so that their relationship was one one-to-many. This required dynamic interfaces with some handler algorithm on them instead of static  form elements, and the server algorithm had to become more complex too. Important challenge moreover that rules regarding which entity belongs to which was declared in ontology extension, and these definition had influenced the interface elements.

During thesis the VIVO idea were further developed, so that the system can cope with the more complex problems, so that only no coding is required for solving the individual cases. To achieve this an extended vocabulary were designed that is able to express the problems of dynamic data input, an the code library were developed for the client and the server that were able to manage the advanced functionality. In the vocabulary related to the data definition the most important advancement is that it is possible to express if a the subject and the object of a particular triple in the graph pattern of the data input process are in one-to-one or in one-to-many relationship with each other (\textit{Triple} vs. \textit{MultiTriple}). On the form definition this cardinality related definition is reflected by the sub forms (\textit{SubformAdder}). These elements allowed the expression of forms and data processor routines that can handle multi dimensional dataset. Further improvement that vocabulary for data definition did not only relate to the RDF triples that were supposed to be stored, but to RDF triples as well which contained described the system, namely the OWL restriction in the ontology extensions (\textit{RestrictionTriple}). The advantage of that definition it connected RDF nodes that were represented on the interface, thus the dependency between form elements could be expressed as well in a declarative way. From the restriction triples the appropriate SPARQL queries are generated and the client and server algorithm through AJAX calls could realize the adaptive interface. Finally, we have seen that the resulting RDF dataset do not necessarily consists of new instance, but existing instance can be selected. The developed vocabulary is able to express these requirements as well, and can allow the convenient finding of the (\texit{AuxNodeSelector}, \textit{InstanceSelector}). 


With the achieved system it is made possible to allow the user to define such data input processes, where the elements of the interface adopts to the ontology extensions, without having to care about how the client and server routines solve that data flow between the user and database. The framework allows the creation of data where the rules lies in ontology extensions. 

This important advantage, because such dynamic data input processes form may contain complex code both on the client and server side. This approach can significantly accelerate the development and allow the Semantic Web based application development for individuals without experience with web application and programming technologies.

\section{Future work}


