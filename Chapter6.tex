

\chapter{Conclusion}

\section{Evaluation}

The three main task of a web application for data management is to allow the user to create, edit and delete particular set of the data. To achieve this several interface elements are required on the web page with the user interacts with. The data manipulation is initiated always by user actions, which trigger different request from the client and user. The task of the server is to interpret the incoming request and perform the necessary data operation in the database. The most important utility of the implemented framework that various data input problems can be defined only by the declaration of the data scheme and the layout of the user interface, and the developer does not have to care about how the client and server routines realizing the data flow between the user and the database. It is achieved such a generic interface routine, which offers the defined input elements and automatically handles the add, edit and delete operations, while the server can interpret the request and based on the data definition is capable of creating, editing and retrieving the data in scope. This approach can significantly accelerate the development and allow the Semantic Web based application development for individuals without experience with web application and programming technologies.

In the preliminaries it was discussed how the VIVO framework allows the declarative definition of simple input process. The added value of the thesis is to allow the shift from the single elements and static graph pattern to a dynamic model which allows the creation of more complex dataset. This is achieved by \textit{MultiTriple} and \textit{SubformAdder} classes in the descriptor logic. The sub form adder allows the definition of a hierarchical form layout for more complex dataset. The data generated by the form




\section{Future work}

