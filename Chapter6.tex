

\chapter{Conclusion}

\section{Evaluation}


The discussed use case requires a relative complex functionality because the individual software agent, the client and the server, participates in order to realize the functionality. and they have to communicate.  The implementation requires both HTML, CSS and JS on the client, and on the server Java and the definition of data in N3 and SPARQL. Their such low level problem like clicking event, ajax call, substitution is eliminated.
This leads to complex codes, which hard to maintain. With the designed vocabulary and framework it is achieved that the development in these programming languages are eliminated and the developer can concentrate on the data related aspects of software. Moreover the software is implemented in a way that it can be easily pluged into an other framework, because the JS can embedded library to any web page, and the server library then in turn just is.

More the most important element of the vocabulary is the \textit{MultiTriple}, which allows a more expressive definition of the scheme of the data. With it we can convert multi dimensional JSON object into RDF and an other way around. This is basically extends SPARQL definition extension. On the client its correspondent is the \textit{SubformAdder}, which allows the definition of same scheme on the interface. With these two element complex datasets can be defined through. Moreover if we consider the capabilies of JavaScript, the user interface model can be extended so that not only the instances can be selected on the floating windows, but maybe other subforms can appear.
This has a potential.
 
Further important part of the vocabulary that supposed to be emphasized is the different restriction triples. With these restriction it is possible to define the dependencies of selectors. But these selectors, dues to the subform adders they contain the options of the elements. Since the implemented JavaScript UI framework is capable of sending request to the server. In such ontological process and system description, nothing is hard coded into HTML, JavaScript or Java code, everything comes loaded from the triple store, regardless if it is class or instance. So they provide a description of the functionality of the form. 


\section{Future work}

