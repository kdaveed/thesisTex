

\chapter{Conclusion}

\section{Evaluation}

The three main task of a web application for data management is to allow the user to create, edit and delete particular set of the data. To achieve this several interface elements are required on the web page with the user interacts with. The data manipulation is initiated always by user actions, which trigger different request from the client and user. The task of the server is to interpret the incoming request and perform the necessary data operation in the database. The most important utility of the implemented framework that various data input problems can be defined only by the declaration of the data scheme and the layout of the user interface, and the developer does not have to care about how the client and server routines realizing the data flow between the user and the database. It is achieved such a generic interface routine, which offers the defined input elements and automatically handles the add, edit and delete operations, while the server can interpret the request and based on the data definition is capable of creating, editing and retrieving the data in scope. This approach can significantly accelerate the development and allow the Semantic Web based application development for individuals without experience with web application and programming technologies.





We have seen that for advanced entry form more complex dataset is required then key-value pairs, becasue they consists of several sub parts. To be able to allow the users for such data creation hierarchical structure is required in the form layout too. This is achieved by the form element \textit{SubFormAdder} which allows the definition of the sub forms. Since scheme data form data scheme follows directly the scheme of the form, this element is considered as data descriptor. On the server side, to define the same structure the \textit{MultiTriple} has been added, that defines how the data has to be processed to generate the RDF data from the submitted multi level JSON object. Moreover it defines the other way around as well, becasue RDF data is queried and processed into the desired form, so that the form algorithm can understand it. 




Further important part of the vocabulary that supposed to be emphasized is the different restriction triples. With these restriction it is possible to define the dependencies of selectors. But these selectors, dues to the subform adders they contain the options of the elements. Since the implemented JavaScript UI framework is capable of sending request to the server. In such ontological process and system description, nothing is hard coded into HTML, JavaScript or Java code, everything comes loaded from the triple store, regardless if it is class or instance. So they provide a description of the functionality of the form. 


\section{Future work}

