

\chapter{Conclusion}

\section{Evaluation}

The three main task of a web application for data management is to allow the user to create, edit and delete particular subset of the data. To realize that both on the client and server side individual algorithms run that communicate with each other. The interface has to offer some data input fields, and buttons through which the user can entry the data and initiate the data operation requests to the server. The task of algorithm on the client is handle the user actions and send the appropriate request, and the other way around show the existing data on the form. The task of the server algorithm is to interpret these request and perform the required data operations. The most important utility of the thesis is that applied vocabulary completely abstracts from these operations, because the user does not have to care about how it is realized that the data flows from the client to the server and other way around, it enough to concentrate only on the scheme of the data and how it appears on the interface. This means a significant simplification, thus the development can be accelerated. 
 





We have seen that for advanced entry form more complex dataset is required then key-value pairs, becasue they consists of several sub parts. To be able to allow the users for such data creation hierarchical structure is required in the form layout too. This is achieved by the form element \textit{SubFormAdder} which allows the definition of the sub forms. Since scheme data form data scheme follows directly the scheme of the form, this element is considered as data descriptor. On the server side, to define the same structure the \textit{MultiTriple} has been added, that defines how the data has to be processed to generate the RDF data from the submitted multi level JSON object. Moreover it defines the other way around as well, becasue RDF data is queried and processed into the desired form, so that the form algorithm can understand it. 




Further important part of the vocabulary that supposed to be emphasized is the different restriction triples. With these restriction it is possible to define the dependencies of selectors. But these selectors, dues to the subform adders they contain the options of the elements. Since the implemented JavaScript UI framework is capable of sending request to the server. In such ontological process and system description, nothing is hard coded into HTML, JavaScript or Java code, everything comes loaded from the triple store, regardless if it is class or instance. So they provide a description of the functionality of the form. 


\section{Future work}

