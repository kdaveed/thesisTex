

\chapter{Conclusion}

\section{Evaluation}




The discussed use case requires a relative complex functionality because the individual software agent, the client and the server, participates in order to realize the functionality, and they have to communicate. The implementation requires both HTML, CSS and JS on the client, and on the server Java and the definition of data in N3 and SPARQL. This leads to complex codes, which hard to maintain. With the designed vocabulary and framework it is achieved that the development in these programming languages are eliminated and the developer can concentrate on the data related aspects of software. Moreover the software is implemented in a way that it can be easily pluged into an other framework, because the JS can embedded library to any web page, and the server library then in turn just is practically the converter from RDF data into JSON, which accesses the databases


----------

Moreover the most important two elements of the vocabulary the \textit{MultiTriple} and \textit{SubformAdder} allow the define the scheme of the data of the input processes and thus the interface structure in a more precise way. Namely it is not the triple pattern is defined, but an additional information about the triples can be set, regarding the cardinalities. Which is basically the conversion of the  


The two most important aspect of the implemented system are the possibility to define the multi dimensional data input represent hierarchical triple structure, which allows more usable web applications that encompasses more functionality on the pages. So a larger set of elements can be edited by one page then by these static forms. The utility of the system that the developer does not have to think of processes, routines, events just in scheme of the data and the constraint between them. And how the elements represent the functionality. 

----------

The multi level input requires complex algorithms in JavaScript and Java. The forms have to be created, and AJAX requests are sent, and on the forms loops are necessary for the processing of the elements. This all eliminated with the \textit{SubFormAdder} elements, and \textit{MultiTriple} for the data definition. The recreation of the for elements is as well an important issue, and thus benefit of the system. Thus the JavaScript library works on the data we do not have to care about the AJAX calls and low level request routines. 

The system can be considered as logic that is an extension of the of the SPARQL language, which does not returns the value in tables but a hierarchical data structure. 


This extends actually the scheme of the 

----------

The restriction triples offer a powerful model to just generate the SPARQL queries.




Allows the conversion of the elements of the ontology extensions into the interface. 


----------






\section{Future work}




