

\chapter{Conclusion}

\section{Evaluation}


To realize multi-level RDF data input, where the elements of the form are dependent on ontology extensions is complex task. As we have seen there are two individual programs, one on the client and one the server, is required to realize the functionality. Their main responsibility is to ensure the data the data flow from the input elements from the user into the database, and the other way around. These are defined in the code in a procedural way by handling data processing, substitution or query result grouping on the server, and on the client JSON object to the form element. This implementation as was addressed requires HTML, CSS, JS and on the server it requires FTL and Java routines. The utility of the implemented framework is able to define the two endpoint of the information flow, namely the form elements and the data scheme in a expressive, way and the developer does not have care about how the JavaScript and Java solves the flow between the elements. Therefore the low level details are hidden and development is accelerated and simplified.

More the most important element of the vocabulary is the \textit{MultiTriple}, which allows a more expressive definition of the scheme of the data. With it we can convert multi dimensional JSON object into RDF and an other way around. This is basically extends SPARQL definition extension. On the client its correspondent is the \textit{SubformAdder}, which allows the definition of same scheme on the interface. With these two element complex datasets can be defined through. Moreover if we consider the capabilies of JavaScript, the user interface model can be extended so that not only the instances can be selected on the floating windows, but maybe other subforms can appear.
This has a potential.
 
Further important part of the vocabulary that supposed to be emphasized is the different restriction triples. With these restriction it is possible to define the dependencies of selectors. But these selectors, dues to the subform adders they contain the options of the elements. Since the implemented JavaScript UI framework is capable of sending request to the server. In such ontological process and system description, nothing is hard coded into HTML, JavaScript or Java code, everything comes loaded from the triple store, regardless if it is class or instance. So they provide a description of the functionality of the form. 


\section{Future work}

