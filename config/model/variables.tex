%layout legacy commands
\renewcommand{\sectionmark}[1]{\markright{\thesection.\ #1}}
\renewcommand{\chaptermark}[1]{\markboth{\thechapter.\ #1}{}}

%user defined commands

% defnisce una pagina bianca con stile plain (migliore delle pagine bianche inserite in automatico dal model book)
\newcommand{\blankpage}{
	\newpage
	% toglie la barra alta dalla pagina vuota
	\thispagestyle{plain}
	% forza una pagina vuota
	\mbox{}
	\newpage
}

%comando per inserire la premessa nel documento (fuori indice)
\newcommand{\premise}[1][]{
	\renewcommand{\theenumi}{#1\roman{enumi}}
	\renewcommand{\labelenumi}{(\theenumi)}
	\thispagestyle{plain}
}


%comandi creati per le convenzioni del documento:
\newcommand{\istage}{\textit{stage}}
\newcommand{\iStage}{\textit{Stage}}
\newcommand{\idfda}{\textit{D.F.D. Assessment System}}
\newcommand{\idfd}{\textit{D.F.D. Consulting}}
\newcommand{\iIT}{\textit{IT}}
\newcommand{\iICT}{\textit{ICT}}
\newcommand{\ibusiness}{\textit{business}}





% indice analitico (comandi con la 'i' scrivono e aggiungono, quelli con 'ii' aggiungono solo, quelli senza niente scrivono solo)
\newcommand{\iASI}{ASI\index{ASI}} %scrive C sharp

%tecnologie
\newcommand{\CSharp}{C♯} %scrive C sharp
\newcommand{\iCSharp}{C♯\index{C♯}} % scrive C sharp e aggiunge l'indice %\unichar{9839}
\newcommand{\idotNET}{.NET\index{.NET}} % scrive .NEt e aggiunge l'indice
\newcommand{\iiWPF}{\index{WPF}} % aggiunge solo l'indice a WPF
\newcommand{\iWPF}{WPF\index{WPF}} % scrive WPF e aggiunge l'indice
\newcommand{\iiWF}{\index{WinForms}}
\newcommand{\iAW}{ANTLRWorks\index{ANTLRWorks}}
\newcommand{\iA}{ANTLR\index{ANTLR}}
\newcommand{\iU}{Unicode\index{Unicode}}
\newcommand{\iPSharp}{\texttt{PdfSharp}\index{PdfSharp@\texttt{PdfSharp}}}
\newcommand{\iTSharp}{\texttt{iTextSharp}\index{iTextSharp@\texttt{iTextSharp}}}

%grammatica
\newcommand{\iiCFG}{\index{CFG (context-free grammar)}}

%programmi
\newcommand{\iVS}{Visual Studio\index{Visual Studio}}
\newcommand{\iVSS}{Visual SourceSafe\index{Visual SourceSafe}}
\newcommand{\iSS}{SQL Server\index{SQL Server}}

% ciclo attivo e passivo
\newcommand{\iCA}{ciclo attivo\index{ciclo!attivo}} % scrive ciclo attivo e aggiunge l'indice
\newcommand{\iCP}{ciclo passivo\index{ciclo!passivo}} % scrive ciclo passivo e aggiunge l'indice
\newcommand{\iiCA}{\index{ciclo!attivo}} % aggiunge l'indice
\newcommand{\iiCP}{\index{ciclo!passivo}} % aggiunge l'indice
\newcommand{\iDCA}{ciclo attivo\index{documento!ciclo attivo}} % scrive ciclo attivo agigunge l'indice al ciclo attivo del documento
\newcommand{\iDCP}{ciclo passivo\index{documento!ciclo passivo}} % scrive ciclo passivo aggiunge l'indice al ciclo passivo del documento
\newcommand{\iiDCA}{\index{documento!ciclo attivo}} % aggiunge l'indice al ciclo attivo del documento
\newcommand{\iiDCP}{\index{documento!ciclo passivo}} % aggiunge l'indice al ciclo passivo del documento
\newcommand{\iGCA}{ciclo attivo\index{gestione!ciclo attivo}} % scrive ciclo attivo e aggiunge l'indice a gestione
\newcommand{\iGCP}{ciclo passivo\index{gestione!ciclo passivo}} % scrive ciclo passivo e aggiunge l'indice a gestione
\newcommand{\iiGCP}{\index{gestione!ciclo passivo}} % aggiunge l'indice a gestione

%componenti
\newcommand{\icMM}{\texttt{MapManager}\index{MapManager@\texttt{MapManager}}}
\newcommand{\iicMM}{\index{MapManager@\texttt{MapManager}}}
\newcommand{\icMF}{\texttt{MapFinder}\index{MapFinder@\texttt{MapFinder}}}
\newcommand{\iicMF}{\index{MapFinder@\texttt{MapFinder}}}
\newcommand{\icPA}{\texttt{PdfAnalyzer}\index{PdfAnalyzer (componente)@\texttt{PdfAnalyzer} (componente)}}

%package e classi
\newcommand{\iPFE}{\texttt{Plain.File.Extraction}\index{Plain.File.Extraction@\texttt{Plain.File.Extraction}}}
\newcommand{\iiPFE}{\index{Plain.File.Extraction@\texttt{Plain.File.Extraction}}}
\newcommand{\iPA}{\texttt{PdfAnalyzer}\index{Plain.File.Extraction@\texttt{Plain.File.Extraction}!PdfAnalyzer (classe)@\texttt{PdfAnalyzer} (classe)}}
\newcommand{\iPP}{\texttt{PdfPage}\index{Plain.File.Extraction@\texttt{Plain.File.Extraction}!PdfPage@\texttt{PdfPage}}}
\newcommand{\iPPar}{\texttt{PdfTextStreamParser}\index{Plain.File.Extraction@\texttt{Plain.File.Extraction}!PdfTextStreamParser@\texttt{PdfTextStreamParser}}}
\newcommand{\iPLex}{\texttt{PdfTextStreamLexer}\index{Plain.File.Extraction@\texttt{Plain.File.Extraction}!PdfTextStreamLexer@\texttt{PdfTextStreamLexer}}}
\newcommand{\iPF}{\texttt{PdfFont}\index{Plain.File.Extraction@\texttt{Plain.File.Extraction}!PdfFont@\texttt{PdfFont}}}
\newcommand{\iPT}{\texttt{PdfText}\index{Plain.File.Extraction@\texttt{Plain.File.Extraction}!PdfText@\texttt{PdfText}}}
\newcommand{\iPC}{\texttt{PdfChar}\index{Plain.File.Extraction@\texttt{Plain.File.Extraction}!PdfChar@\texttt{PdfChar}}}

%parti di un PDF
\newcommand{\iH}{\texttt{Header}\index{PDF!Header@\texttt{Header}}}
\newcommand{\iFT}{\texttt{File Trailer}\index{PDF!File Trailer@\texttt{File Trailer}}}
\newcommand{\iCRTable}{\texttt{Cross Reference Table}\index{PDF!Cross Reference Table@\texttt{Cross Reference Table}}}
\newcommand{\iB}{\texttt{Body}\index{PDF!Body@\texttt{Body}}}

% stream
\newcommand{\iS}{stream\index{stream}}
\newcommand{\iiS}{\index{stream}}
\newcommand{\iCS}{stream\index{content stream}}
\newcommand{\iiCS}{\index{content stream}}

%fasi dello stage
\newcommand{\iifS}{\index{studio del dominio!fase di}}
\newcommand{\iifA}{\index{analisi!fase di}}
\newcommand{\iifP}{\index{progettazione!fase di}}
\newcommand{\iifC}{\index{codifica!fase di}}
\newcommand{\iifV}{\index{verifica e validazione!fase di}}
\newcommand{\iifD}{\index{documentazione!fase di}}

%attività dello stage
\newcommand{\iiaS}{\index{studio del dominio!attività di}}
\newcommand{\iiaA}{\index{analisi!attività di}}
\newcommand{\iiaP}{\index{progettazione!attività di}}
\newcommand{\iiaC}{\index{codifica!attività di}}
\newcommand{\iiaV}{\index{verifica e validazione!attività di}}
\newcommand{\iiaD}{\index{documentazione!attività di}}

%matrici
\newcommand{\Tm}{$T_{m}$}
\newcommand{\Tlm}{$T_{lm}$}

%peratori
\newcommand{\Tc}{\texttt{Tc}\index{operatore!di stato}}
\newcommand{\Tw}{\texttt{Tw}\index{operatore!di stato}}
\newcommand{\Tz}{\texttt{Tz}\index{operatore!di stato}}
\newcommand{\TL}{\texttt{TL}\index{operatore!di stato}}
\newcommand{\Tf}{\texttt{Tf}\index{operatore!di stato}}
\newcommand{\Tr}{\texttt{Tr}\index{operatore!di stato}}
\newcommand{\Ts}{\texttt{Ts}\index{operatore!di stato}}

\newcommand{\Td}{\texttt{Td}\index{operatore!di posizionamento}}
\newcommand{\Tmm}{\texttt{Tm}\index{operatore!di posizionamento}}
\newcommand{\Tstar}{\texttt{T*}\index{operatore!di posizionamento}}

\newcommand{\Tj}{\texttt{Tj}\index{operatore!di stampa}}
\newcommand{\Tquote}{\texttt{'}\index{operatore!di stampa}}
\newcommand{\Tdblquote}{\texttt{\textquotedbl}\index{operatore!di stampa}}
\newcommand{\TJ}{\texttt{TJ}\index{operatore!di stampa}}

\def \baseDir {/Users/konkolydavid/Dropbox/Thesis/Images/}

\def \II {\baseDir2_Preliminaries/}
\def \III {\baseDir3_Problem_Statement/}

\newcommand{\img}[3]{
	\begin{figure}[h]
		\makebox[\textwidth][c]{\includegraphics[width=\textwidth]{#1#2}}%
		\caption{#3}
		\label{fig:key}
\end{figure}}

\newcommand{\html}[3]{\textless#1 #2 \textgreater #3 \textless#1\textgreater}

\newcommand{\prefix}[2]{
	\begin{center}
	  @prefix #1: \textless#2\textgreater .
	\end{center}
}

\newcommand{\cent}[1]{
 \begin{center}
	#1
 \end{center}
}

\newcommand{\boldcenter}[1]{
 \begin{center}
 	\textbf{#1}
 \end{center}
}

\newcommand{\itcent}[1]{
	\begin{center}
		\textit{#1}
	\end{center}
}

\newcommand{\sparql}[2]{
	\begin{lstlisting}[captionpos=b, caption=#1, label=lst:sparql,
	basicstyle=\ttfamily,frame=single]
	#2
	\end{lstlisting}
}