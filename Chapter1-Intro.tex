\chapter{Introduction}
\pagenumbering{arabic}%DO NOT REMOVE THIS


\section{RDFBones Project}

The master thesis is written in the frame of the project called \textit{RDFBones}. The project is conducted by the \textit{Biological Anthropology} department of the \textit{Universitätsklinikum Freiburg}, and funded by the \textit{Deutsche Forschung Gemeinschaft} (DFG). The main idea of the project is to make possible the definition of the rules and steps of particular anthropological investigations in a machine-readable way. This is achieved by the development of a core ontology in Ontology Web Language (OWL), that describes the general scheme of the processes, while custom problems are defined by means of so-called ontology extensions. The extensions are represented by further ontological RDF statements, which has the advantage that a web application for creating RDF data about the execution of the processes, can programmed in generic way so that it adapts its interface to the various cases. During the project an open-source web application framework, called VIVO, is adopted and developed.

\section{Goal of the thesis}

Data creation about research processes happens through multiple different pages, where each page is responsible for a particular subset of data. This means the general scheme of the input forms is characterized by the subset of the core ontology for which the data has to be created on them. The idea of the thesis is develop a web application framework that is capable of generating the its interfaces based on a declarative definition of the dataset they supposed to create. To achieve this a descriptor language is designed which is capable of expressing the dataset and the how it is mapped to the interface. The utility of the idea is that the developed system can be applied not only for the conscise solution of the problem of the \textit{RDFBones} project without coding, but for any kind domains, where the rules are defined through ontology extensions.



\section{Thesis outline}

The second chapter contains the background information necessary to understand the problems the thesis aims to solve. The first two subsections handles the RDF data model and the ontologies applied in the project, while the third section is about the basics of the web applications and the VIVO framework. The third chapter presents the problem statement and the brief explanation of the scheme of the solution. The fourth chapter presents the elements of designed data model for the web application domain, and present how they can be used to solve data input problems of the \textit{RDFBones} project. The fifth chapter is dedicated to the discussion of the developed software framework. Its first section provides an overview about the main components of the software, while the second give deeper insight into implementation. The last, sixth chapter covers the conclusion and the evaluation of the achieved system, and present the further potential in the idea.
