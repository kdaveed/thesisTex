\chapter{Introduction}
\pagenumbering{arabic}%DO NOT REMOVE THIS


\section{RDFBones Project}

The master thesis is written in the frame of the project called \textit{RDFBones}. The project is conducted by the Biological Anthropology department of the \textit{Universitätsklinikum Freiburg}, and funded by the \textit{Deutsche Forschung Gemeinschaft} (DFG). 
The goal of the project is to develop a data standard and web application to document research activity related to biological anthropology. Withing the project RDF data model is used, it has the advantage against relational data model that it is easier to extend. The idea is to develop a core ontology, which describes scheme of the investigations generally, and define so-called ontology extensions that describes the particular cases. The task of the web application is adopt the web user interface to the extensions. In this way the developed system is applicable for various investigations. \cite{infrastructure}

\section{Goal of the thesis}

Creating data by web applications happens so through web pages. In our cases the user interface does not consist only of static data input fields (i.e textarea, checkbox), but dynamic elements for more complex cases. Through the user interaction with page a particular dataset is created, and it is sent to server where it is processed and stored in a database. This means within a web application there are two individual software agents communicating with each other. The idea of the thesis is to design a data model that is capable of expressing layout of the web user interface, and as well as the scheme of the dataset that has to be created, and develop general purpose application library both for the client and the server side, that operates based this high-level definition. By means of this approach, it is achieved that the various use-cases of the project can be solved by creating a descriptor dataset of designed data model. The implemented software is integrated into the applied open-source RDF data based web application framework called VIVO.

\section{Thesis outline}

The second chapter contains the background information, that is necessary to understand the problem the thesis aims to solve. The first two subsection handles the RDF data model and the ontologies applied in the project, while the third section is about the basics of the web applications. 

The third chapter presents the problem statement. The first section discusses the scheme of the data that has to be created to document anthropological research, while the second section show what does it mean in terms of the web application programming. 

The fourth chapter presents the elements of designed data model, and shows how they can be used to solve data input problems, through the use-cases of use-cases of the \textit{RDFBones} project. 


The fourth and the fifth chapter are describing the proposed solution for the problem. The fourth chapter outlines the higher-level modeling of the application, and fifth in turn describes the how the implemented framework can operate upon the declarative definition.

The last, sixth chapter covers the conclusion and the evaluation of the achieved system, and present the further potential in the idea. \figref{outline} illustrates the structure of the thesis, where the blue denotes the chapter with conceptual content, and the yellow in turn the ones with practical content.
