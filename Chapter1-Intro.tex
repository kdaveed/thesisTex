\chapter{Introduction}
\pagenumbering{arabic}%DO NOT REMOVE THIS


\section{RDFBones Project}

The master thesis is written in the frame of the project called \textit{RDFBones}. The project is conducted by the Biological Anthropology department of the Universitätsklinikum Freiburg, and funded by the Deutsche Forschung Gemeinschaft (DFG). The main idea of the project is to make possible the definition of the rules and steps of particular anthropological investigations in a machine-readable way. This is achieved by the development of a core ontology in Ontology Web Language (OWL), that describes the general scheme of the processes, while custom problems are defined by means of so-called ontology extensions. The extensions are represented by further ontological RDF statements, which has the advantage that a web application for creating RDF data about the execution of the processes, can programmed in generic way so that it adapts its interface to the various cases. During the project an open-source web application framework, called VIVO, is adopted and developed.

\section{Goal of the thesis}

Data creation about research processes happens through multiple different web pages, where each page is responsible for a particular subset of the data. The main structure of the individual input forms is characterized by the scheme of the data they create, which is in our case a subset of the core ontology. While the elements of the pages in turn are defined in the certain extensions of the addressed scheme. The idea of the thesis is develop a web application framework that is capable of generating its interfaces based on a declarative definition of the dataset they supposed to create. To achieve this, a descriptor language is designed which is capable of expressing RDF data models  and their mapping to the interface. The utility of the idea is that the developed system can be applied not only for the concise solution of the problems of the \textit{RDFBones} project, but for any kind of domain, where the rules are defined by means of ontology extensions.


\section{Thesis outline}

The second chapter conveys all the background information that is necessary to understand the problem solved by the thesis. The first two subsections handle the RDF data model and the ontologies applied in the project, while the third section is about the basics of the web application technologies and the VIVO framework. The third chapter contains the problem statement and explains briefly the scheme of the proposed solution. The fourth chapter presents the elements of designed descriptor language, and demonstrates how it can be applied to specify two use-cases of the \textit{RDFBones} project. The fifth chapter is dedicated to the discussion of the functionality of the developed software framework. Its first section provides an overview about the main components of the software, while the second gives a deeper insight into the implementation. The last, sixth chapter covers the conclusion and the evaluation of the achieved system, and presents the further potential in the idea.
