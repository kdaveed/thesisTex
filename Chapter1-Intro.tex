\chapter{Introduction}
\pagenumbering{arabic}%DO NOT REMOVE THIS


\section{RDFBones Project}

The master thesis is written in the frame of the project called \textit{RDFBones}. The project is conducted by the Biological Anthropology department of the \textit{Universitätsklinikum Freiburg}, and funded by the \textit{Deutsche Forschung Gemeinschaft} (DFG).  The goal of the project is to develop a data standard and web application to document research activity related to biological anthropology. Withing the project RDF data model is used, it has the advantage against relational data model that it is easier to extend. The idea is to develop a core ontology, which describes scheme of the investigations generally, and define so-called ontology extensions that describes the particular cases. The task of the web application is adopt the web user interface to the extensions. In this way the developed system is applicable for various investigations. \cite{infrastructure}

\section{Goal of the thesis}

Creating data by web applications happens so through web pages. In our cases the user interface does not consist only of static data input fields (i.e textarea, checkbox), but dynamic elements for more complex cases. Through the user interaction with page a particular dataset is created, and it is sent to server where it is processed and stored in a database. This means within a web application there are two individual software agents communicating with each other. The idea of the thesis is to design a data model that is capable of expressing layout of the web user interface, and as well as the scheme of the dataset that has to be created, and develop general purpose application library both for the client and the server side, that operates based this high-level definition. By means of this approach, it is achieved that the various use-cases can be solved by creating a descriptor dataset of designed data model. The implemented software is integrated into the applied open-source RDF data based web application framework called VIVO.

\section{Thesis outline}

The second chapter contains the background information necessary to understand the problems the thesis aims to solve. The first two subsections handles the RDF data model and the ontologies applied in the project, while the third section is about the basics of the web applications and the VIVO framework. The third chapter presents the problem statement and the brief explanation of the scheme of the solution. The fourth chapter presents the elements of designed data model for the web application domain, and present how they can be used to solve data input problems of the \textit{RDFBones} project. The fifth chapter is dedicated to the discussion of the developed software framework. Its first section provides an overview about the main components of the software, while the second give deeper insight into implementation. The last, sixth chapter covers the conclusion and the evaluation of the achieved system, and present the further potential in the idea.
