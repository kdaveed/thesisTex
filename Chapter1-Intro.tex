\chapter{Introduction}
\pagenumbering{arabic}%DO NOT REMOVE THIS


\section{RDFBones Project}

The master thesis is written in the frame of the project called \textit{RDFBones}. The goal of the project is to develop a data standard, as well a web application for documenting research activity related to biological anthropology. In anthropology each institute has different set of skeletal remains and they conduct their investigations differently, thus it is not possible to cover during the project period the problem of other institute as well. Therefore the core idea of the project is that to make possible the definition of custom processes through ontology extensions. This is only possible by means of RDF data, because the ontologies in OWL can be extended, and specific rules can be defined through restrictions.
We are using Semantic Web application framework called \textit{VIVO}, which is capable of editing not only the RDF data itself but the ontology as well. So the idea of the project that not only the data will be exchangeable but the extensions as well.
The task of the application is to generate the data input forms based on the ontology extensions. 

Web application so that they do not 



\section{Goal of the thesis}


The data input forms and the data model is tightly coupled. Which means that the application has limited tasks of exercises. The 
The goal is to implement a such vocabulary that allows abstracts from the low implementation. This would allow the developers 
the implementation of data input processes of people without programming experience. This

The vocabulary incorporates the definition of the input form layout, and the underlying data structure. Based on this compact definition
the framework has to be able to generate the interfaces and perform the necessary data operations.
T

\section{Thesis outline}

