\chapter{Introduction}
\pagenumbering{arabic}%DO NOT REMOVE THIS


\section{RDFBones Project}

The master thesis is written in the frame of the project called \textit{RDFBones}. The project is conducted by the \textit{Biological Anthropology} department of the \textit{Universitätsklinikum Freiburg}, and funded by the \textit{Deutsche Forschung Gemeinschaft} (DFG). The main idea of the project is to make possible the definition of the rules and steps of particular anthropological investigations in a machine-readable way. This is achieved by the development of a core ontology in Ontology Web Language (OWL), that describes the general scheme of the processes, while custom problems are defined by means of so-called ontology extensions. The extensions are represented by further ontological RDF statements, which has the advantage that a web application for creating RDF data about the execution of the processes, can programmed in generic way so that it adapts its interface to the various cases. During the project an open-source web application framework, called VIVO, is adopted and developed.

\section{Goal of the thesis}

The goal of the thesis is to solve the data input problems of the \textit{RDFBones} project in a way which can be applied for any kind of problems, where the rules of the domain lies in RDF statements. Data creation for a research process happens through multiple different pages, where each page is responsible for a particular set of data. The most fundamental challenge is that the elements of the data input forms are dependent ontology extensions. The idea of the thesis is to define the pages only by describing the scheme of the dataset they create, which addresses the elements of the core ontology. 


how this dataset is mapped to the input fields of form. 






\section{Thesis outline}

The second chapter contains the background information necessary to understand the problems the thesis aims to solve. The first two subsections handles the RDF data model and the ontologies applied in the project, while the third section is about the basics of the web applications and the VIVO framework. The third chapter presents the problem statement and the brief explanation of the scheme of the solution. The fourth chapter presents the elements of designed data model for the web application domain, and present how they can be used to solve data input problems of the \textit{RDFBones} project. The fifth chapter is dedicated to the discussion of the developed software framework. Its first section provides an overview about the main components of the software, while the second give deeper insight into implementation. The last, sixth chapter covers the conclusion and the evaluation of the achieved system, and present the further potential in the idea.
