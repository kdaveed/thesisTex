\chapter{Introduction}
\pagenumbering{arabic}%DO NOT REMOVE THIS


\section{RDFBones Project}

The master thesis is written in the frame of the project called \textit{RDFBones}. The goal of the project is to develop a data standard, as well a web application for documenting research activity related to biological anthropology. The challenge is that in anthropology each institute has different set of skeletal remains and have different research interest and scopes. So to achieve that the developed application can be used by various cases it has to extensible. For this reason RDF data model is applied, since it is more suitable for such purposes than relational data model.\cite{infrastructure}
Building web application is large effort, therefore the software is not developed from scratch, but an existing Semantic Web application framework, called VIVO is used. The advantage of VIVO that the it has a capability to adopt his interface to the ontology, which is highly desirable feature for our project. 


\section{Goal of the thesis}





\section{Thesis outline}

The second chapter contains the background information, that is necessary to understand the problem the thesis aims to solve. In the first subsection handles the RDF data model and the ontologies applied in the project, while the third section is about the basics of the web applications.

The third chapter presents the problem statement. The first section discusses the scheme of the data that has to be created to document anthropological research, while the second section show what does it mean in terms of the web application programming. 

The fourth and the fifth chapter are describing the proposed solution for the problem. The fourth chapter outlines the higher-level modeling of the application, and fifth in turn describes the how the implemented framework can operate upon the declarative definition.

The last, sixth chapter covers the conclusion and the evaluation of the achieved system, and present the further potential in the idea. \figref{outline} 


\img{\I}{Outline.png}{Structure of the chapters}{outline}

