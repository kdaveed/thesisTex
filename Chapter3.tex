\chapter{Problem Statement}


\section{Multi level data input}


In the previous chapter it has been discussed how can we implement simple data input forms within the VIVO framework. The simplicity of the illustrated problems lied in that the number of instances which were created through the forms was constant, in particular one, and only their types and literal attributes were set by the user through HTML input elements. Nevertheless there are such entities that are more complex, and they consist of several sub part, that are represented in the ontology with further classes. Such as a skeletal subdivision consists of bone organs, or a study design execution consists of assays, which consist of input bones and output data. Consequently the RDF dataset by such entities incorporates a set of instances organized into a tree structure (\figref{ps1}, where the green node is that main instance). 

\img{\III}{ps1.png}{RDF triples for complex entities}{ps1}


This means in terms of the data input form that it has to offer such interface elements which allows the user to add the components step by step. 







\img{\III}{ps2.png}{Multi level form}{ps2}



\begin{lstlisting}[basicstyle=\footnotesize, frame=single, caption={Multi level form data in JSON}, label=multiData, captionpos=b, belowskip=1em, aboveskip=2em]
{	type : "eq:elementA",
	label : "Element_4391",
	components : [
		{ type : "eq:componentA1",
		  label : "Component_8531",	
		  subComponents : [ { ... } ],
		},
		{ ... }
	]
}
\end{lstlisting}


\img{\III}{ps3.png}{Ontology}{ps3}


\section{Solution scheme}

