\chapter{Problem Statement}

The first section discusses the problem general way by addressing the challenges of the implementation both for the client and the server side in case of more elaborate data input processes. The second section in turn refers to the use-cases of the \textit{RDFBones} project and to the limitations of VIVO framework. Finally, the third part introduces the main elements the descriptor logic and gives and insight into the functionality of the developed system. 

\section{Multi level data input}

In the previous chapter it has been discussed how can we implement simple data input forms within the VIVO framework. The simplicity of the illustrated problem lied in that the number of instances which were created through the process was constant, in particular one, and only a set of literal attributes were set by the user through HTML input elements. Nevertheless there are more complex entities consisting of several sub parts, where these sub parts are represented in the ontology with further classes. Consequently the RDF dataset for such entities incorporates multiple instances organized into a tree structure. \figref{ps1} shows and example ontology and an RDF dataset. The classes (ellipses) without notation are subclasses of the three main classes and their names are not relevant.

\img{\III}{ps1.png}{Ontology and RDF triples for complex entities}{ps1}

Such dataset poses the requirement for the input form, that the user has to be offered such interface elements which enable to add the components and subcomponents step by step. Adding a component means in terms of the form that a new sub form appears which contains further input fields for the component instance.

\img{\III}{ps2.png}{Multi level form}{ps2}

\figref{ps2} shows the layout of the form for multi level data. The additional element compared to the static HTML form is the field with an button for adding the sub forms. The dotted rectangle stands for the container element that encompasses the added sub forms. The form data contains the same way the key value pairs for the main form element, but it has an additional array field that contains the data objects of the sub forms. The sub form data object works the same way, if the have sub forms then they contains further arrays. To realize such functionality, JavaScript routine is required on the form, that ads the sub form elements to the container, and generates the appropriate form data upon the user actions.

Listing \ref{multiData} shows the JSON object generated during by form from \figref{ps2}, where the objects are surrounded with ("\{\}"), while the arrays with ("[]"). After the submission the server has to process this object by iterating through the arrays of them, and generate the appropriate RDF triples.

\begin{lstlisting}[basicstyle=\footnotesize, frame=single, caption={Multi level form data in JSON}, label=multiData, captionpos=b, belowskip=1em, aboveskip=2em]
{	type : "eq:elementA",
	label : "Element_4391",
	components : [
		{ type : "eq:componentA1",
		  label : "Component_8531",	
		  subComponents : [ { ... } ],
		},{ 
			... 
		}]
}
\end{lstlisting}

Further challenge is that the options of the type selector on the sub forms, are dependent on the selected type of the parent form (the form which the sub form has been added from). This means that the client has to load asynchronously the values, by sending AJAX request containing the selected type value to server. The task of the server is to perform the query that retrieves the classes defined through restrictions in the ontology. The goal of this functionality is firstly to ensure that only such data is created that conforms to the rules defined in the ontology, secondly the interface is much more usable if not all the components are listed, just the ones that are belong to the selected element. Moreover in this way the validation on the server side after submission can be omitted.

Finally, really important part of the problem is the editing of the existing data. By editing the application has to restore the state of the form, in which it has been submitted by the data creation. In the previous chapter we have seen that it is currently solved by defining  SPARQL queries that retrieves the form variables, which are set to the input forms through the \textit{Freemarker} variables. But since the form data is in this case not just a set of key value-pairs but a multi level data object, this approach is not sufficient. An algorithm is required that generates the multi level JSON object from the existing triples iteratively. Furthermore after the arrival of the existing data to the client, an other routine has to reset the state of the form with the appropriate sub forms and certain selector options as well.

\section{RDFBones use-cases in VIVO}

During the project two use-cases and their solutions are discussed. We have seen in the previous chapter is the data input process definition for various cases happens through the data definition of RDF triples with variable (graph pattern) and the the form layout. \figref{rdfBones1} and \figref{rdfBones2} illustrate the graph pattern and the multi level form layout of the two most important cases. The yellow rectangles denotes the main nodes of the dataset, which were characterized solely by literals in the static case.

\img{\III}{rdfBones1.png}{Skeletal subdivision graph pattern and form layout}{rdfBones1}

\img{\III}{rdfBones2.png}{Investigation graph pattern and form layout}{rdfBones2}

The problem is that it would be necessary to define custom JavaScript routines for each cases, which adds the sub forms and handles the dependency between the particular selector fields. Moreover since VIVO cannot process the multi level JSON object coming from the client after the submission based on this single triple pattern in string, for each individual Java routines have to be written that creates the data, as well as for the retrieval. Furthermore the individual variables in the triple pattern do not appear only once in the result dataset, thus their value cannot be defined with SPARQL a single queries.

\section{Solution scheme}

In VIVO the data input problems were solved by means of assigning values to a set of fields of a certain Java class, based on which the generic algorithm creates and retrieves the data. The task of the thesis is to design a vocabulary, that allows the expression of multi level cases. The vocabulary is a set of Java classes, whose instances serve as the definition of the problems. Descriptor objects are processed and further Java and JSON objects are generated based on the Java and JavaScript libraries can manage the functionality.

In the triple pattern definition the most fundamental improvement is that it is not stored as a string but as a set of different types of triple. The three main types of triple are the \textit{Triple}, the \textit{MultiTriple} and \textit{RestrictionTriple}. The \textit{Triple} has the same role as a triple in the string in VIVO. The \textit{MultiTriple} allows the expression of the hierarchy in the graph pattern, which presents on the form as well. With the \textit{MultiTriple} it is possible to prepare the submission handler routine that a set of variables are not appear as single values in the form data, within further objects in an array. While the \textit{RestrictionTriple} is used to express dependencies between classes in the graph pattern. This definition is used to generate the SPARQL queries that return the options of the dependent variables. \figref{solutionScheme} depicts the extended graph pattern definition, where the restriction triple are depicted with red arrow, and the multi triple with double line arrow.

\img{\III}{solScheme.png}{Extended RDF graph pattern definition}{solutionScheme}

For the form layout definition the two main classes are the \textit{Form} and \textit{FormElement}. There are subclasses of the \textit{FormElement} class, which represent the different input fields. The form element that defines the multi level case is represented by the \textit{SubformAdder} class, which incorporates the elements of the sub form naturally. The hierarchical Java object set is converted into JSON object, based on which the JavaScript library is able to operate. 

\figref{frameworkFunctionality} functionality depicts the scheme of the functionality. The most important improvement wrt. to VIVO that there is processor that is capable of inferring some addition descriptor data object from the original descriptor objects. Such as SPARQL queries for the data dependencies and data retrieval.  

\img{\III}{332_1.png}{Framework functionality outline}{frameworkFunctionality}
