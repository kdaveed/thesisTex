\chapter{Problem Statement}


In section (ontologies) in particular in the 2.2.3 \textit{RDFBones} ontology scheme of the dataset that represents anthropological research activity has been discussed, while the section 2.3.2 introduced the basics of RDF data input. 


\section{Multi level data input}


The previous chapter showed simple cases of data input problems, where the values of the HTML form elements were substituted into a predefined set of RDF triples. In the presented cases the simplicity lies in that the number of the created RDF instances was constant, and only their types and attributes were set by the user. This meant that it was sufficient to send a request consisted only of such key-value pairs. Nevertheless there are more complicated cases where the set of instances created during the data input process are defined by the user. This requires such web interfaces that allow the user to add form elements dynamically and can create the appropriate form data.



\img{\III}{ps1.png}{Multi level data}{ps1}



\img{\III}{ps2.png}{Multi level form}{ps2}



\begin{lstlisting}[basicstyle=\footnotesize, frame=single, caption={Multi level form data in JSON}, label=multiData, captionpos=b, belowskip=1em, aboveskip=2em]
{	type : "eq:elementA",
	label : "Element_4391",
	components : [
		{ type : "eq:componentA1",
		  label : "Component_8531",	
		  subComponents : [ { ... } ],
		},
		{ ... }
	]
}
\end{lstlisting}


\img{\III}{ps3.png}{Ontology}{ps3}


\section{Solution scheme}

