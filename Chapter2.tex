\chapter{Preliminaries}

\section{Web applications}

\section{Semantic Web}

\subsection{RDF}

In RDF, abbreviation for Resource Description Framework, the information of the web is represented by means of triples. Each triple consists of a subject, predicate and object. The set of triples constitute to an RDF graph, where the subject and object of the triples are the nodes, the predicates are the edges of the graph.
An RDF triple is called as well statement, which asserts that there is a relationship defined by the predicate, between subject and the object. The subjects and the objects are RDF resources. A resource can be either an IRI (Internationalized Resource Identifier) or a literal or a blank node (discussed later). A resource represents any physical or abstract entity, while literals hold data values like string, integer or datum. 
Basically there are two types of triples, the one that links two entities to each other, and the other that links a literal to an entity. The former expresses a relationship between two entities, and the latter in turn assign an attribute to the entity.
Common practice is to represent IRI with the notation prefix:suffix, where the prefix represents the namespace, and the expression means the concatenation of the namespace denoted by the prefix, with the suffix. This convention makes the RDF document more readable. The namespace of RDF is the http://www.w3.org/1999/02/22-rdf-syntax-ns\#, whose prefix is in most cases "rdf".   This is defined on the following way:


\prefix{rdf}{http://www.w3.org/1999/02/22-rdf-syntax-ns\#}

Literals are strings consisting of two elements. The first is the lexical form, which is the actual value, and the second is the data type IRI.  RDF uses the data types from XML schema. The prefix (commonly xsd) is the following :

\prefix{xsd}{http://www.w3.org/2001/XMLSchema\#}

So a literal value in RDF looks as follows:

\cent{"Some literal value"ˆˆxsd:string}

The RDF vocabulary provides some built-in IRIs. The two most important are, the rdf:type property, and the rdf:Property class. The meaning of the triples, where the predicate is the property rdf:type  is that the subject IRI is the instance of the class denoted by the object. Therefore the following statement holds in the RDF vocabulary:

\cent{rdf:type 	rdf:type	rdf:Property.}

It is maybe confusing that an IRI appears in a triple as subject and predicate as well, but we will see by the RDFS vocabulary that it is inevitable to express rules of the language.  
To be able to represent information about a certain domain, it is necessary to extend the RDF vocabulary with properties and classes. The classes will be discussed in the next section, but here it is explained how custom properties can be defined. The namespace of the example is the following: 

\prefix{eg}{http://example.org\#}


The example dataset intends to express information about people, which university they attend and how old are them. To achieve this two properties are needed:

\cent{eq:attends		rdf:type		rdf:Property .\\
	eq:age			rdf:type		rdf:Property .}


The actual data about a person:

\cent{
	eg:JanKlein		eq:attends		eq:UniversityOfFreiburg . \\
	eg:JanKlein		eq:age			"21"ˆˆxsd:integer . }


\subsection{RDF Schema}


The previous section gave an insight into RDF world by showing how can information stored by means of triples. However the explanation did not mention that each RDF dataset has to have scheme, which is also called ontology. The ontology describes the set of properties and classes and how are they are related to each other. RDFS provides a mechanism to define such ontologies using RDF triples. The most important elements of the RDFS vocabulary can be seen on the following image. 

\imgK{\II}{12_rdfs_vocabulary.png}{Main structure of the RDFS vocabulary}

The two most important classes in the RDFS vocabulary is the rdfs:Class and the rdfs:Resource. The rdfs:Class is class, because it is the instance of itself, and the same way the rdfs:Resource is a class. The rdf:Propery and the rdfs:Literal are both classes as well. The rdfs:domain, rdfs:range, rdfs:subPropertyOf and rdfs:subClassOf are properties. Important to note that these properties are subjects and predicates in the same time in the RDFS vocabulary graph. Also they describe themselves like rdf:type.
The properties rdfs:domain and rdfs:range describe for the property the type of the subject and object respectively, which with it can build a triple as predicate. The following image illustrates their meaning:

\imgK{\II}{13_rdfs_domain_range.png}{RDFS domain and range definition}

Since both the rdfs:domain and rdfs:range are properties themselves, they have as well their domain and range, which is the class rdfs:Resource. 
The property rdfs:subClassOf expresses subclass relationship between classes. It means if a class B is a subclass of class A, and resource R is the instance of class B, then resource R is the instance of class A as well. Since it describes the relationship between two classes its both domain and range is the class rdfs:Class. 
The property rdf:subPropertyOf expresses the relationship between two properties. If property P2 is sub property of P1 and two instances are related by P2 then they are related by P1 as well. Its domain and range is the class  rdf:Property.
Now everything is given to define the ontology for the example of the previous section.

\imgK{\II}{14_rdfs_ex_ontology.png}{RDFS domain and range definition}


\subsection{OWL}

OWL, abbreviation for Ontology Web Language is an extension of the RDFS vocabulary. OWL allows expressing additional constraints on the data, above the range and domain definitions. These constraints are called restrictions. Restrictions are conventionally expressed by blank nodes. Blank nodes do not have IRIs, but it is defined through the triples in which they participate as a subject. For example a restriction stating that the instances of the class eg:FootballTeam can build a triple through the eg:hasPlayer property only with the instances of eg:FootballPlayer class can be expressed the following way:

\begin{lstlisting}[basicstyle=\footnotesize, captionpos=b, caption=OWL restriction in N3 format, label=lst:sparql,
frame=single]
eg:FootballTeam rdfs:subClassOf  [ 
	rdf:type		owl:Restriction ; 
	owl:onProperty		eg:hasPlayer ; 
	owl:allValuesFrom	eg:FootballPlayer .
]
\end{lstlisting}

owl:Restriction is class and owl:onProperty and owl:allValuesFrom are properties. It can be seen that class, on which the restriction applies is the subclass of the restriction blank node.
Furthermore OWL is capable of expressing qualified cardinality restriction. For example the statement that a basketball team has to have exactly five players, look as follows in OWL:


\begin{lstlisting}[captionpos=b, caption=OWL restriction in N3 format, label=lst:sparql,
basicstyle=\footnotesize,frame=single]
PREFIX rdf: <http://www.w3.org/1999/02/22-rdf-syntax-ns#>
PREFIX eg: <http://example.org> 

eg:BasketballTeam rdfs:subClassOf  [
	rdf:type	owl:Restriction ; 
	owl:onProperty	eg:hasPlayer ; 
	owl:onClass	  eg:Player ; 
	owl:qualifiedCardinality "5"(*@\textasciicircum@*)(*@\textasciicircum@*)xsd:nonnegativeInteger
] . 

\end{lstlisting}


These two examples cover the thesis related features of OWL. The next image depicts the OWL vocabulary. 

\imgK{\II}{15_owl_vocab.png}{A subset of OWL vocabulary}


There are two new class types are the owl:Class and the owl:Restriction. The rdf:Property has two subclasses, the owl:ObjectProperty and owl:DataTypeProperty. owl:ObjectProperty represent the properties that links instances to instances, and the owl:DataTypeProperty is those that link instances to literals. The following two images shows the domain and range definitions of the OWL properties used to describe restrictions.

\imgK{\II}{16_object_properties.png}{OWL object properties}
\imgK{\II}{17_qualified_card.png}{Properties for qualified cardinalities}

\subsection{SPARQL}

SPARQL is a query language for querying data in RDF graphs. A SPARQL query is a definition of a graph pattern through variables and constants. The following example query returns all IRIs that represent a football player: 

\begin{lstlisting}[captionpos=b, caption=SPARQL Query I., label=lst:sparql,
basicstyle=\footnotesize,frame=single]
SELECT ?player  
WHERE { 
	?player		rdf:type	eg:FootballPlayer .	
}
\end{lstlisting}

In the example the query consist of only one triple. The subject is a variable and the predicate and the object are constant. Therefore the triple store looks all the triples and checks the predicate is rdf:type and the object is eg:FootballPlayer. 
It is well possible to not just ask the IRI of the players but further information by adding additional triples to the query in order to ask the name for example of the player:

\begin{lstlisting}[captionpos=b, caption=SPARQL Query II., label=2nd:sparql,
basicstyle=\footnotesize,frame=single]
SELECT ?player ?name 
WHERE { 
	?player		rdf:type	 eg:FootballPlayer .	
	?player		eg:name		 ?name .
}
\end{lstlisting}

The result table in this case will contain two columns, one with the IRI of the person and one with their name. Important that it is as well possible to query blank nodes by introducing a variable for it. So if we want to list all the instances that are coming into question as player to a football team we can formulate the following query:

\begin{lstlisting}[captionpos=b, caption=SPARQL Query III., label=3rd:sparql,
basicstyle=\footnotesize,frame=single]
SELECT ?person ?name
WHERE {
	eg:FootballTeam	rdfs:subClassOf		?restriction .
	?restriction		rdf:type					owl:Restriction .
	?restriction		owl:onProperty		eg:hasPlayer .
	?restriction		owl:allValuesFrom	?playerType .
	?player					rdf:type					?playerType .
	?player					eg:name						?name . 
}
\end{lstlisting}


\section{Applied Ontologies}

Ontologies are used to describe types, relationships and properties of objects of a certain domain. It is a common practice to used already defined 
ontologies rather developing an own. The first reason is, that the development of an ontology is a complex and a tedious process, and requires a lot of resource. Secondly, it is reasonable to use standardized vocabularies, in order to make data from same domain but different sources inter-operable.


\subsection{Foundational Model of Anatomy - \textit{FMA}} 

The foundational Model of Anatomy ontology is an open source ontology written in OWL. FMA is a fundamental knowledge source for all biomedical domains, and it provides a declarative definition of concepts and relationships of the human body for knowledge based applications. It contains more than 70 000 classes, and 168 different relationships, and organize its entities into a deep subclass tree \cite{Rosse2003478}. All types of anatomical entities are represented in FMA, like molecules, cells, tissues, muscles and of course bones. In our project we use only the subset of the FMA. The taken elements are the subclasses of the following two classes and the three properties: 

\begin{itemize}
	\item{Classes}
		\subitem \textit{Subdivision of skeletal system} - fma:85544 
		\subitem \textit{Bone Organ} – fma:5018
\end{itemize}

\begin{itemize}
	\item{Properties}
		\subitem \textit{fma:systemic\_part\_of}
		\subitem \textit{fma:constitutional\_part\_of}
		\subitem \textit{regional\_part\_of}
\end{itemize}

The class \textit{Bone Organ} is the superclass of all bones in the human skeleton. Each bone belong to a skeletal subdivision and a skeletal subdivision can be a part of another skeletal subdivision. This relationship in both cases is expressed by the property \textit{fma:systemic\_part\_of}. To define which bone organ belongs to which skeletal subdivision FMA uses OWL restrictions (see \figref{fma1}). The properties \textit{fma:constitutional\_part\_of} and \textit{fma:regional\_part\_of} \textit{fma:constitutional\_part\_of} are discussed later.

\img{\III}{1ontologyStructure.png}{Ontology structure for skeleton}{fma1}

Finally the advantage of using the FMA ontology is that, if in the future further elements of the human body have to be addressed by the research processes, i.e. muscles, then these classes can be easily integrated to the currently applied subset.

\subsection{Ontology for Biomedical Investigations - \textit{OBI}} 


The aim of OBI ontology, is to provide the formal representation of the biomedical investigation in order to standardize the processes among different research communities. It is a result of a collaborative effort of several working groups, and it continuously evolving as new research methods are being developed. Its main function to describe the rules how biological and medical investigations have to be performed. OBI reuses terms from BFO \textit{Basic Formal Ontology} IAO \textit{Information Artifact Ontology} and  OBO \textit{Open Biological and Biomedical Ontologies}\cite{10.1371/journal.pone.0154556}. To define processes OBI uses the following three general classes:

\begin{itemize}
	\item  \textit{Information Content Entity} - obo:IAO\_0000030
	\item  \textit{Material Entity} - obo:BFO\_0000040
	\item  \textit{Process} - obo:BFO\_0000015
\end{itemize}

Information Content Entity represent results of a specific measurement, while Material Entity stands for the objects, on which the meausurements have been performed. 
The Process could mean any kind of step within an investigation, from the planning, through execution till the conclusion.

\begin{itemize}
	\item  \textit{Planning} - obo:OBI\_0000339
	\item  \textit{Study Design Execution} - obo:OBI\_0000471
	\item  \textit{Drawing a conclusion} - obo:OBI\_0000338
\end{itemize}


In our project the following three properties are used:

\begin{itemize}
	\item  \textit{has part} - obo:BFO\_00000051
	\item  \textit{has specified input} - obo:OBI\_00000293
	\item  \textit{has specified output} - obo:OBI\_00000299
\end{itemize}


\section{Web applications}

\subsection{Client-sever architecture}

Usually web applications do not consist of one single page, but of several different pages. In order to navigate between the pages of the application, the HTML document contains links that trigger further HTTP requests. Links in HTML can be defined by means of the <a/> tag.  The most important parameter of this tag is href, whose value contains the URL of the HTTP request. 
Let assume that an application’s main page is accessible through the URL http://newsPortal.com. Common practice that subpages of the application can be called through various url-mappings, which means the main URL is extended with a keyword that denotes the page to be requested. 

\imgK{\II}{1_client_server.png}{Client server communication}

Usually web applications do not consist of one single page, but of several different pages. In order to navigate between the pages of the application, the HTML document contains links that trigger further HTTP requests. Links in HTML can be defined by means of the <a/> tag.  The most important parameter of this tag is href, whose value contains the URL of the HTTP request. 
Let assume that an application’s main page is accessible through the URL http://newsPortal.com. Common practice that subpages of the application can be called through various url-mappings, which means the main URL is extended with a keyword that denotes the page to be requested. 

\imgK{\II}{2_http_request.png}{HTML Link}

The link in Figure 2 shows the link for the subpage. Programming the server incorporates the task of assignment of the url-mappings to particular classes, which are responsible for the response preparation. In Java web application these responder classes are called servlets an the definition of the mapping-class assignment looks as follows:

\imgK{\II}{3_servlet_mapping.png}{Servlet mapping definition in web.xml}

A modern web application do not just send static web pages to the client, that contains in this case the political articles, but the articles are stored in a database, and the pages are generated dynamically by substituting the retrieved data into so-called template files. 
First of all the task of responding requires a query that retrieves that data from the database. By applications using relational data model, the tables and attributes are always modeled by classes of the used object oriented programming (OOP) language. So the data retrieval is the instantiation of the classes in scope. 

\imgK{\II}{4_data_flow.png}{Data flow}

In our example the news are stored in the NEWS table and the application has a class named News with the same attributes that the table has.  Consequently one single row of the table can be stored in an instance of the News class. As the database returns a table with multiple news, thus the resulting Java data will have the type List<News>. Then this in this simple case the list is passed to the template engine together with the template file. 

\imgK{\II}{5_template_file.png}{Template file}

The template file is a description of how the data has to be converted into HTML document. It can be seen that it is possible for instance to declare a list on the input variable newsList. Then the template engine iterates through the News objects and by accessing its fields (title, summary, id) and generates the HTML for each element. 

The template file contains the following link

\begin{center}
	\scriptsize{\html{a}{href="http://newsPortal.com/wholeNews?id=\${news.id}"}{Read more}}
\end{center}

which redirects to the page where the whole article can be seen. To achieve this it is necessary to equip each link with the parameter id that holds the ID of NEWS item, order to inform the server about which article’s whole text has to be displayed. Then the servlet class of the mapping /wholeNews has to perform the following query where the id is the input.

\imgK{\II}{6_sql_query.png}{SQL query with parameter}

Web applications do not only just display existing data, but they allow the users to enter their new data. In HTML the element used for data input is called form. Form is a container, and it consists of particular form elements according to the data to be added. 

\imgK{\II}{7_form_layout_html.png}{Form layout and HTML document}

Submitting the form to the server send an HTTP request with multiple parameters, where they are divided through the \& character.

\imgK{\II}{8_request_parameter.png}{Request with parameters}

By the data entry creation the task of the controller is to get the values from the request an instantiate the class representing the data to be created. Then initialized class instance is passed to the database where the entered data will be persistently stored.

\imgK{\II}{9_example_java.png}{Example Java routine for data storage}


\subsection{JavaScript bacics}

JavaScript (JS) is the programming language of the web browser. A JS code can be embedded into any HTML document between <script></script> tags. The most fundament capability of JS, is that it is capable of manipulating the elements of the web page. The following example illustrates a simple case, where clicking a button can change the page by adding a new div to an other div.

\imgK{\II}{10_simple_js_example.png}{Simple JavaScript example}

The HTML page contains with two divs with the id-s button and container. JavaScript handles each element on the page as objects. These objects can be referenced by \$(“\#id”) where the id is the id parameter of the html tag. Thus the definition of a click event to the first div is done by writing the following code:

\boldcenter{\$(“\#button”).click( …   the handler function …)}

to the script. The added function defines only one single operation, which uses the append function on the \$(“\#container”) div object. The input parameter is a new div object, created by JS with the text value Element. 


\subsection{Data in JS}

In JavaScript there are not typed variable, each type variable is defined by the keyword var. In the simplest case a JavaScript variable holds a single value of a  primitive type like boolean, string or number (a). Or it be an array (b) or an object (c), which is a set of key-value pairs. The following figure shows the definition of three main JavaScript variable types.


\begin{lstlisting}[basicstyle=\footnotesize, frame=single, caption={JavaScript data types}, captionpos=b]
var a = "someText"			//Data field
var b = ["element", 3.45, false]	//Array
var c = {				//JSON Object
	name : "Bob",
	friends : ["Jack", "John"],
	studies : {
		university : "University of Freiburg",
		course : "Computer Science"
	}		
}
\end{lstlisting}

Important to note the values of JavaScript object (c) can be single values, arrays or even further objects. The access of these variables are quite simple.

\begin{lstlisting}[basicstyle=\footnotesize, frame=single, caption={JavaScript data access}, captionpos=b]
	b[2]  -->   3.45
	c.studies.course --> "Computer Science"
\end{lstlisting}


These variables can be set either through the template variables like any value in the HTML document. Or the event handler routines can assign values to the variables based on user actions.

\subsection{AJAX}

AJAX is an abbreviation for Asynchronous JavaScript And XML. This is a technology that allows the web browser to exchange data with the server without reloading the whole page. AJAX calls are initiated from JavaScript and of course JS itself is responsible for handling the response. This mechanism is where the JavaScript data plays a really important role. The following example shows and AJAX based solution for loading the whole text of an article.

\imgK{\II}{11_ajax_load.png}{Loading new element through AJAX}

The example from the image illustrates extends the previous case so that the click function contains the AJAX call. This call is practically the same as basic request from the <a/> tag in HTML. It has a URL and a data object. The data object in this example consists of only one key-value pair, with the key newsID. The value is the JavaScript variable id, whose own value was set at the beginning of the script part by \${news.id} template variable. This is the way that Java variables can be passed the JavaScript variables.  The done function of the AJAX routine defines what has to be done with the data that arrives. The response data coming from the server is accessible in the msg variable. In the example we assume the server return only the string of the whole text, which will be set as the text of the new div.




\section{VIVO Framework}

VIVO is an open source web application framework, developed particularly for browsing and editing RDF data. VIVO utilizes that the data scheme in RDF is stored by means of triples as well, and it can adopt its pages to the ontology.  It offers an ontology editor and there are particular features of the application that can be customized through a specific configuration dataset.  This dataset is in RDF too, and describes the way in which the data is displayed and can be edited on the web pages. VIVO allows to manipulate this configuration triples via the web interface, which enables the extension of the application to some extent conveniently without any coding. Finally there is a possibility to import any RDF file to VIVO’s triple store.

\subsection{Class groups}

One important feature of the VIVO framework is the possibility to order the classes of the ontology into so-called class groups. If a class is assigned to a class group then it appears in the list of the admin panel, which is used to select the type of the new instance.

\imgK{\II}{18_document_class_group.png}{Document class group on the admin panel}

Further possibility of class groups that it is possible to create links on the main menu (can be seen on the top of \figref{classGroupPage}), which redirects to a page where all the instances are listed that belong to one of the classes of the class group.

\img{\II}{19_vivo_class_group.png}{ VIVO class group page for documents}{classGroupPage}

The application configuration showed in the last two images is  defined by the set of configuration triples on \figref{rdfConfigData}.

\img{\II}{20_rdf_config_data.png}{ RDF configuration data in VIVO}{rdfConfigData}

There are three instances of the classes display:NavigationElement, display:ClassGroupPage and for vitro:ClassGroup. The triples itself are self-explanatory, but important to note that property vitro:inClassGroup is the one that connects the configuration dataset to the domain ontology.  

\subsection{Profile Pages}

A profile pages in VIVO displays information about a particular RDF instance. These pages can be reached by clicking one of the entries of the list on the class group pages. The profile page organizes data connected to the individual into tabs. Each tab displays the properties of a specific property group. \figref{vivoProfilePage} shows a screenshot from the profile page under the Overview tab.

\img{\II}{21_profile_page.png}{VIVO profile page layout}{vivoProfilePage}

As it was already addressed this page adopts to the ontology by querying the properties whose domain or range is the type of the instance to display.  The properties bibo:author and vivo:description are assigned to the property group overview, and they appear on the page only because the both have the domain bibo:Document class.

\img{\II}{22_profile_page_config.png}{Triples contributing to the displayed profile page layout}{configTriples}

An additional important feature of VIVO is the definition of so-called faux properties. They are really similar to the rdfs:subProperty, but they are considered by the profile pages.

\subsection{Default Data Entry Forms}

On \figref{configTriples}, next to the predicate labels (authors, description) there are plus image elements, which are a links. These links redirect the user to data entry forms where new triple can be added. 

\img{\II}{23_http_request.png}{HTTP request for data entry form}{httpEntryForm}

They initiate the HTTP request depicted on \figref{httpEntryForm}. The server gets with which subject and predicate the triple has to be created.

The subject of the triple is the instance; from whose the profile page the request has been initiated. The predicate is the property to which the link belongs. The data entry forms allows the user to set the object of this triple. By the property bibo:author the domain is the class foaf:Person, thus application offers each existing instance of this class to select, or allows to add a new instance as an object. 

\imgK{\II}{24_object_prop_entry_form.png}{Object property entry form for bibo:author}

In the case of the property vivo:description, the domain is the class rdfs:Literal thus the entry form displays a text input field.

\imgK{\II}{25_data_prop_entry_form.png}{Data property entry form}


\subsection{Custom Entry Forms} \label{vivoCef}

VIVO allows the editing of the triples through default entry forms only one by one. However it is often the case that it desired to add multiple triples, thus larger dataset by one entry form. This is as well possible in VIVO through custom entry form definition.
Let assume an entry form, which let the user add new publications to person instance. About the publication its title, abstract and the date of publishing can be stored. The left part of following image shows dataset of the example. The red nodes denote the variables that are coming as input from the entry form; the green means that its value has to be an unused IRI, and the grey stands for constants. 

\imgK{\II}{26_data_def.png}{Data definition graphical (left) and lexical (right)}{dataDef}

The variable ?subject is the instance from whose profile page entry form was called.  To declare the information held by the graphical representation of the triples from \figref{dataDef}, three static Java variables are needed. Two arrays of string for the inputs and new resources (literalsFromRequest, newResources) and a string for the triples (tripleString).  Moreover the configuration class has an additional variable for defining the template file for the form layout (VIVO uses Freemarker template engine, and the .ftl extension stands for Freemarker Template File). 
The last step towards the definition of the custom entry form is to connect the property eg:Published with the predicate vitro:customEntryFormAnnotation to the literal value that holds the name of the entry form configuration class.

\imgK{\II}{27_cef_class_def.png}{Definition of custom entry form configuration class }


