\begin{itemize}	
	\item{Validation - cyclic graph - multiple dependencies}
	\item{subgraph subform dependencies}
	\item{Descriptor for data and processing}
\end{itemize}

\chapter{Framework functionality} \label{5}


The aim of this chapter is to present the main mechanisms of the implemented software framework, which is capable of operating on high-level configuration data. Section \ref{51} contains a more abstract description of the functionality, while section \ref{52} goes into the implementation details both of the server and client side programming, including how the framework can be integrated into the applied web application. In both sections the explanation refers to the examples discussed in the previous chapter.


\section{Main software modules and tasks} \label{51}


In \figref{332_1} we have seen the main work flow and components of the framework. \figref{scheme_1} is in turn a more detailed depiction of the software modules and processes of the application.


\img{\V}{scheme_1.png}{More detailed scheme}{scheme_1}


This section is divided into three subsections. First is the part (section \ref{511}) discusses the  processing of the configuration data and generation of the functionality descriptor objects for the client and the server. The second part (section \ref{512}) is about the client functionality including the asynchronous communication with the server. Finally the third (section \ref{513}) part covers the process after the submission, namely how the RDF data is generated based on the data coming from the client, as well as how the existing data is retrieved from the triples store.


\subsection{Validation} \label{511}


The processor algorithm has four main tasks to solve. The validation of the configuration data, generation of the form descriptor JSON object, and the Java object for data dependencies and for  the graph model. 


\img{\V}{processor_tasks.png}{Processor tasks}{processor_tasks}


The input of the algorithm is the set of triples describing the data model with its constraints, and form model, which refers to the nodes in the triple set. The first task to do is the validation because the descriptors are not supposed to be generated based on incorrect configuration data. The validator process has three scope of the checking, the nodes, the graph and the form.
 
\figref{valid_1} depicts an example data model and illustrates the cases of valid and invalid nodes (\figref{graph_notation} contains the meaning of the shapes and colors). The explanation starts with the discussion of the form input nodes. Node \textit{2} is valid because it is possible to generate a SPARQL query that retrieves the possible values of it. The query contains one triple which ask the subclasses of the constant class. Furthermore node \textit{4} is valid as well, because there is path to it from a valid class node, therefore for there is again a SPARQL query for its values. However the variable \textit{3} is not valid, because it does not contribute to any triple in path with valid input node or constant. Here it is important to note that the path cannot come from the instance to the class, just the other way around. So the path 2 --> 1 --> 4 counts in the processor routine, but 2 --> 5 --> 6 --> 3 does not.
  
\img{\V}{valid_1.png}{Valid and invalid nodes}{valid_1}


The next task regarding the nodes is to check if each has a value by the RDF triple creation. The instances coming from the interface are automatically valid, because their URI is an input value. But the ones that have to generated newly and get as value a new unused URI from the triple store, must constribute to triple as subject, where the predicate is \textit{rdf:type} and the object is a valid class. For this reason the node \textit{5} is valid, since its type class is valid, but node \textit{6} is not. Moreover node \textit{7} is not valid as well, since it does not have any type class defined in the data model. Finally regarding the literals the validation is the simplest, either they appear on the form or have constant value, otherwise they are invalid. 


\img{\V}{valid_2.png}{Valid and invalid graph}{valid_2}


Above the nodes important itself the whole graph built by the triples have to be investigated. In the previous chapter the triple type \textit{MultiTriple} were introduced. The rule regarding this type of triple that it divides the graph into subgraphs, and the subgraphs can connect to each other only be these triples. \figref{valid_2} illustrates the valid and invalid graph arrangements. The reason is that only to this type of scheme can the JSON object of the created by the data input process be mapped.


\subsection{Dependencies and form functionality}  \label{512}


The set of triples describing the data model builds a graph where the various input nodes are connected to each other. Further task of the processor algorithm to determine the subgraphs that define the SPARQL queries for the node appear on the form. The elements on the form has a specified order, and the rule that have to be considered during the processing, is that a variable can be dependent only of the main input variables or such variables that are before it on the form. The reason is that the dependency is practically SPARQL query with one output and with one or more inputs, and input node have to available by the execution of the query.

\img{\V}{form.png}{Form element order}{form}

\figref{form} shows the simplified form layout of the skeletal inventories and emphasizes the order of the elements with the number. The idea is that values of the selector for the node \textit{subSubdivision} can be only dependent on the main input nodes because it is the first node. The \textit{boneOrgan} in turn can be dependent on the \textit{subSubDivision} too because its value has to be set before the options of the subform is loaded, thus it can be substituted into a SPARQL query.


\img{\V}{skeletalInventory.png}{Skeletal inventory data constraints}{skeletalInventory}

\figref{skeletalInventory} depicts the scheme of the data constraints of the skeletal inventory, based on which the dependency retriever algorithm can be exemplified. As it was mentioned the task is to get one or more path from the variable of to an other node whose value is available for it.  As the form's first element refers to the node \textit{subDivision} its dependency has to be evaluted first. Since the node \textit{subSubDivision} contributes to two restriction triples in the data scheme it is necessary to check the both paths. Since this node cannot be dependent on any other form node, it is dependent on the two main input nodes (\textit{subjectUri} and \textit{rangeUri}). While the by dependency for the node \textit{boneOrgan} the \textit{rangeUri}
does not count, since the algorithm terminates already by the \textit{subSubdivison} variable. \figref{dependency} shows the results subgraphs of the algrithm, where the output is depicted with green and the inputs with red and light red respectively.

\img{\V}{dependencyResult.png}{Form dependency subgraphs}{dependency}


This result of the processor concerns both the descriptor of the client, and the server. On the server the depencies are stored a classes that have the necessary fields, for the inputs and output and for the triples. Then these initialized dependency descriptor class objects are stored in a data map, where the key is the variable name of the output node. This map is a field of the form configuration class, thus if the client asks for the appropriate form variable through AJAX these SPARQL query assemble by the triples for can be executed with the incoming values.


For the client the dependency description is just the an assignment of an array, which contains the input node of the dependency, to the form node. In the case of the \textit{subSubdivision} it is an empty array, because it is not dependent on any form element, but for example the \textit{boneOrgan} in the use-case for the study design execution the is dependent from the \textit{assayType} and the skeletal inventory. The task of the form by loading new sub form is to check this array and get the variables required values from the form. This mechanism will be discussed in more detail in section \ref{522}. 

Above the dependency the descriptor the JavaScript framework obtains the descriptor data file for the form elements upon which it can generate the form. The mechanism of the form description retrieval a quite simple because it is practically the convestion of a Java object into a JSON object. The fields of the Java object appear in the JSON objects as key. In 

\begin{lstlisting}[basicstyle=\footnotesize, frame=single, caption={Java to JSON}, captionpos=b]
public JSONObject getDescriptor(){
	JSONObject object = new JSONObject();
	object.put("title", this.title)
	... 
	return object;	
}
\end{lstlisting}

Where the \textit{this.title} has contains the value for the title of the form object in the descriptor Java object. The same way if the Java class has list field, like the form has list of form elements, then getDescriptor() routines of each element is called and inserted into a JSON array. Moreover if the form element is sub form adder then it has a field \textit{subForm}. This comes as well of course into the descriptor, and this is the way how the multi level JSON is created.


\begin{lstlisting}[basicstyle=\footnotesize, frame=single, caption={Subform descriptor}, captionpos=b]
	object.put("subForm", this.subForm.getDescriptor());
\end{lstlisting}


\figref{formDescriptor} illustrates the generated JSON object for a form with the data dependencies too. The task of the JavaScript routine to interpret this configuration data and generate the form and subform upon user action.

\img{\V}{formDescriptor.png}{Form descriptor JSON}{formDescriptor}



\subsection{Graph model generation}  \label{513}

Above the form generation form descriptor and the dependencies the output of the processor algorithm is the graph model that represent the data of the form. The task of this graph model is to generate the data coming from the form submission, and make it available for


Like by the form it is as well a hierarchical data structure. It starts with the graph model.



\img{\V}{graphData.png}{Main fields of the graph}{graphData}

\img{\V}{graphProcess.png}{Main fields of the graph}{graphProcess}


\section{Implementation} \label{52}


This request arrives to the server and based on the parameter \textit{predicateUri} the application finds the configuration triples. More details about how this parameter value is assigned to the configuration setting will be provided in the next chapter.

As it was described in the previous chapter the whole process starts from the pages for data display with the HTTP requests initiated by the links of the data fields. 



\subsection{Server side} 

\begin{itemize}
	\item This section introduces how application on the server is able to operate based on data queried from the configuration dataset
	
\end{itemize}

\subsubsection{Overview}

\begin{itemize}
	\item The following image shows the most important steps of the data input process
\end{itemize}

\imgK{\IVII}{overview.png}{Overview of the mechanism on the server}


\subsubsection{Form representation in Java}

\imgK{\IVII}{uml1.png}{UML Diagram of the classes for the form}

\begin{itemize}
	\item The literal fields asks for the type of the variable it represents and the type of the descriptor will be based on this literal field.
\end{itemize}


\imgK{\IVII}{formDescriptor.png}{Form descriptor JSON object}

\subsubsection{Data model in Java}


\begin{itemize}
	\item Querying the configuration triples regarding statements
	\item Processing into the tree structured graph model
\end{itemize}


\imgK{\IVII}{graphUML.png}{Graph UML}

Validation:

\begin{itemize}
	\item Data dependencies can be over graphs
	\item But graphs can be connected only through multi statements
	\item The graph has to be a tree. There are no use cases right now where any loop would be required
\end{itemize}

\imgK{\IVII}{jsonVSGraph.png}{JSON vs graph model}

\begin{itemize}
	\item Data saving mechanism explanation
	\item Data retrieval mechanism explanation
\end{itemize}

\subsubsection{Data Dependencies}


\begin{itemize}
	\item The data dependencies are important only for the form. 
	\item Everything starts with the form descriptor.
	\item There are cases where from the main form no element appears on the form.
	\item A have to bring examples to some problems that illustrate the problem.
	\item Here comes at first the ontology awareness into question....
\end{itemize}


\imgK{\IVII}{exampleProblem.png}{Example problem}

where,

\imgK{\IVII}{explanation.png}{Elements of the simplified notation}

\imgK{\IVII}{restrictionDirection.png}{Two restriction directions}

\begin{itemize}
	\item Post processing for different restriction types
\end{itemize}


\imgK{\IVII}{variableDependency.png}{Getting dependent data}

\subsubsection{Editing and deleting data}

\begin{itemize}
	\item This feature was not introduced on the overview image but it is really important.
\end{itemize}


\imgK{\IVII}{edit.png}{Difference between the submissions and edit data on the form}


\subsection{Client side}

\begin{itemize}
	\item In this chapter it is discussed how the forms are implemented in JavaScript
	\item Each subsection contains code examples that gradually introduce the functionalities
	\item Codes are mostly simplified to facilitate the understanding		
\end{itemize}


\subsubsection{Object oriented JavaScript}

\begin{itemize}
	\item The task of code in on the interface is to operate based on a configuration data dynamically.
	\item There two subtasks, the generation of the UI elements (i.e input fields, buttons, etc.) and manage the data input and display on the form
	\item To solve these problem, an object oriented approach is applied
	\item This means that there are classes that handles both the UI and the data related tasks
	\item See an example for the class definition in JS
\end{itemize}


\begin{lstlisting}[basicstyle=\footnotesize, frame=single, caption={JavaScript class}, captionpos=b]
class StringField {
constructor(...){
this.container = $("<div/>")
...
}
someMethod(){...}			
}
\end{lstlisting}

\begin{itemize}
	\item Each form elements are represented by such objects, and the form loading based on the descriptor runs by the initialisation of these elements.  
\end{itemize}

\begin{lstlisting}[basicstyle=\footnotesize, frame=single, caption={Form generation based on configuration data}, captionpos=b]
var formData = new Object()
for(var i = 0, i < formElements.length; i++){
var descriptor = formElements[i]
var element = null
switch(descriptor.type){
case "stringField":
element = new StringField(descriptor, formData)
break;
case "..." : 
}
$("#formContainer").append(element.container)
}	
\end{lstlisting}


%\imgK{\IVIII}{1_1_oopJS.png}{UML Class diagram for the JS implementation}
%\imgK{\IVIII}{1_2_layoutOfClasses.png}{DOM Elements of the classes}

\begin{itemize}
	\item Explanation of the code ...
	\item This is the way how it is possible to generate interfaces
\end{itemize}


\subsubsection{Handling data}

\begin{itemize}
	\item Above the element generation for the different forms it is necessary to handle of course the data based on the descriptor
	\item Each form element's descriptor contains a field called dataKey. The value of this field will be the key of data in the form data object.
\end{itemize}


\begin{lstlisting}[basicstyle=\footnotesize, frame=single, caption={Data saving}, captionpos=b]
class StringField {
constructor(descriptor, formData){
this.descriptor = descritor
this.formData = formData
this.inputField = $("<input/>").change(this.handler)
}

handler(){
this.formData[this.descriptor.dataKey]=this.inputField.val() 
}			
}
\end{lstlisting}



\begin{itemize}
	\item The previous code illustrates how the form element object set the global form data field based on configuration data
	\item Further details about the code...
\end{itemize}

\begin{itemize}
	\item Handling existing data
	\item It can be the case that form is loaded for editing. Then in this case the formData variable coming as input to the constructor contains the value for the dataKey? 
\end{itemize}

\begin{lstlisting}[basicstyle=\footnotesize, frame=single, caption={Data saving}, captionpos=b]
class StringField {
constructor(descriptor, formData){
if(formData[descriptor.dataKey] != undefined){
this.editMode = true
}
}

handler(){
if(this.editMode){
var oldValue = this.formData[this.descriptor.dataKey]
var newValue = this.inputField.val() 
}
AJAX.updateField(oldValue, newValue)
}			
}
\end{lstlisting}


\subsubsection{Sub form adders}


\begin{itemize}
	\item Descriptor of the sub form adder contains a field called subform
	\item Then the initialisation of the form happend through the Form class.	
	\item This is used by the initial form loading as well
\end{itemize}

\begin{lstlisting}[basicstyle=\footnotesize, frame=single, caption={Sub form adder routine}, captionpos=b]
class SubformAdder {
constructor(descriptor, formData){
...
this.addButton = $("<div/>").text("Add").click(this.add)    
this.subFormDescriptor = this.descriptor.subForm
this.formData[this.descriptor.dataKey] = []
}

add(){
var subformDataObject = new Object()
this.formData[this.descriptor.dataKey].
push(subFormDataObject) 
this.subFormContainer.append(
new Form(this.subFormDescriptor, subFormDataObject))
}
}
\end{lstlisting}

\begin{itemize}
	\item Important for each subform a new JSON object is generated which is pushed to the array		
	\item The same way the subform adder checks if the this.formData[this.descriptor.dataKey] contains already existing data and adds them if they are there
\end{itemize}

\subsection{Data dependency}

\subsection{Form validation}
