\begin{itemize}	
	\item{Validation - cyclic graph - multiple dependencies}
	\item{subgraph subform dependencies}
	\item{Descriptor for data and processing}
\end{itemize}

\chapter{Framework functionality} \label{5}


The aim of this chapter is to present the main mechanisms of the implemented software framework, which is capable of operating on high-level configuration data. Section \ref{51} contains a more abstract description of the functionality, while section \ref{52} goes into the implementation details both of the server and client side programming, including how the framework can be integrated into the applied web application. In both sections the explanation refers to the examples discussed in the previous chapter.


\section{Main software modules and tasks} \label{51}


In \figref{332_1} we have seen the main work flow and components of the framework. \figref{scheme_1} is in turn a more detailed depiction of the software modules and processes of the application.


\img{\V}{scheme_1.png}{More detailed scheme}{scheme_1}


This section is divided into three subsections. First is the part (section \ref{511}) discusses the  processing of the configuration data and generation of the functionality descriptor objects for the client and the server. The second part (section \ref{512}) is about the client functionality including the asynchronous communication with the server. Finally the third (section \ref{513}) part covers the process after the submission, namely how the RDF data is generated based on the data coming from the client, as well as how the existing data is retrieved from the triples store.


\subsection{Validation} \label{511}


The processor algorithm has four main tasks to solve. The validation of the configuration data, generation of the form descriptor JSON object, and the Java object for data dependencies and for  the graph model. 


\img{\V}{processor_tasks.png}{Processor tasks}{processor_tasks}


The input of the algorithm is the set of triples describing the data model with its constraints, and form model, which refers to the nodes in the triple set. The first task to do is the validation because the descriptors are not supposed to be generated based on incorrect configuration data. The validator process has three scope of the checking, the nodes, the graph and the form.
 
\figref{valid_1} depicts an example data model and illustrates the cases of valid and invalid nodes (\figref{graph_notation} contains the meaning of the shapes and colors). The explanation starts with the discussion of the form input nodes. Node \textit{2} is valid because it is possible to generate a SPARQL query that retrieves the possible values of it. The query contains one triple which ask the subclasses of the constant class. Furthermore node \textit{4} is valid as well, because there is path to it from a valid class node, therefore for there is again a SPARQL query for its values. However the variable \textit{3} is not valid, because it does not contribute to any triple in path with valid input node or constant. Here it is important to note that the path cannot come from the instance to the class, just the other way around. So the path 2 --> 1 --> 4 counts in the processor routine, but 2 --> 5 --> 6 --> 3 does not.
  
\img{\V}{valid_1.png}{Valid and invalid nodes}{valid_1}


The next task regarding the nodes is to check if each has a value by the RDF triple creation. The instances coming from the interface are automatically valid, because their URI is an input value. But the ones that have to generated newly and get as value a new unused URI from the triple store, must constribute to triple as subject, where the predicate is \textit{rdf:type} and the object is a valid class. For this reason the node \textit{5} is valid, since its type class is valid, but node \textit{6} is not. Moreover node \textit{7} is not valid as well, since it does not have any type class defined in the data model. Finally regarding the literals the validation is the simplest, either they appear on the form or have constant value, otherwise they are invalid. 


\img{\V}{valid_2.png}{Valid and invalid graph}{valid_2}


Above the nodes important itself the whole graph built by the triples have to be investigated. In the previous chapter the triple type \textit{MultiTriple} were introduced. The rule regarding this type of triple that it divides the graph into subgraphs, and the subgraphs can connect to each other only be these triples. \figref{valid_2} illustrates the valid and invalid graph arrangements. The reason is that only to this type of scheme can the JSON object of the created by the data input process be mapped.


\subsection{Dependencies and form functionality}  \label{512}


The set of triples describing the data model builds a graph where the various input nodes are connected to each other. Further task of the processor algorithm to determine the subgraphs that define the SPARQL queries for the node appear on the form. The elements on the form has a specified order, and the rule that have to be considered during the processing, is that a variable can be dependent only of the main input variables or such variables that are before it on the form. The reason is that the dependency is practically SPARQL query with one output and with one or more inputs, and input node have to available by the execution of the query.

\img{\V}{form.png}{Form element order}{form}

\figref{form} shows the simplified form layout of the skeletal inventories and emphasizes the order of the elements with the number. The idea is that values of the selector for the node \textit{subSubdivision} can be only dependent on the main input nodes because it is the first node. The \textit{boneOrgan} in turn can be dependent on the \textit{subSubDivision} too because its value has to be set before the options of the subform is loaded, thus it can be substituted into a SPARQL query.


\img{\V}{skeletalInventory.png}{Skeletal inventory data constraints}{skeletalInventory}

\figref{skeletalInventory} depicts the scheme of the data constraints of the skeletal inventory, based on which the dependency retriever algorithm can be exemplified. As it was mentioned the task is to get one or more path from the variable of to an other node whose value is available for it.  As the form's first element refers to the node \textit{subDivision} its dependency has to be evaluted first. Since the node \textit{subSubDivision} contributes to two restriction triples in the data scheme it is necessary to check the both paths. Since this node cannot be dependent on any other form node, it is dependent on the two main input nodes (\textit{subjectUri} and \textit{rangeUri}). While the by dependency for the node \textit{boneOrgan} the \textit{rangeUri}
does not count, since the algorithm terminates already by the \textit{subSubdivison} variable. \figref{dependency} shows the results subgraphs of the algrithm, where the output is depicted with green and the inputs with red and light red respectively.

\img{\V}{dependencyResult.png}{Form dependency subgraphs}{dependency}


This result of the processor concerns both the descriptor of the client, and the server. On the server the depencies are stored a classes that have the necessary fields, for the inputs and output and for the triples. Then these initialized dependency descriptor class objects are stored in a data map, where the key is the variable name of the output node. This map is a field of the form configuration class, thus if the client asks for the appropriate form variable through AJAX these SPARQL query assemble by the triples for can be executed with the incoming values.


For the client the dependency description is just the an assignment of an array, which contains the input node of the dependency, to the form node. In the case of the \textit{subSubdivision} it is an empty array, because it is not dependent on any form element, but for example the \textit{boneOrgan} in the use-case for the study design execution the is dependent from the \textit{assayType} and the skeletal inventory. The task of the form by loading new sub form is to check this array and get the variables required values from the form. This mechanism will be discussed in more detail in section \ref{522}. 

Above the dependency the descriptor the JavaScript framework obtains the descriptor data file for the form elements upon which it can generate the form. The mechanism of the form description retrieval a quite simple because it is practically the convestion of a Java object into a JSON object. The fields of the Java object appear in the JSON objects as key. In 

\begin{lstlisting}[basicstyle=\footnotesize, frame=single, caption={Java to JSON}, captionpos=b, belowskip=1em, aboveskip=2em]
public JSONObject getDescriptor(){
	JSONObject object = new JSONObject();
	object.put("title", this.title)
	... 
	return object;	
}
\end{lstlisting}

Where the \textit{this.title} has contains the value for the title of the form object in the descriptor Java object. The same way if the Java class has list field, like the form has list of form elements, then getDescriptor() routines of each element is called and inserted into a JSON array. Moreover if the form element is sub form adder then it has a field \textit{subForm}. This comes as well of course into the descriptor, and this is the way how the multi level JSON is created.


\begin{lstlisting}[basicstyle=\footnotesize, frame=single, caption={Subform descriptor}, captionpos=b, belowskip=1em, aboveskip=2em]
	object.put("subForm", this.subForm.getDescriptor());
\end{lstlisting}


\figref{formDescriptor} illustrates the generated JSON object for a form with the data dependencies too. The task of the JavaScript routine to interpret this configuration data and generate the form and subform upon user action.

\img{\V}{formDescriptor.png}{Form descriptor JSON}{formDescriptor}



\subsection{Graph model generation}  \label{513}

Above the form generation form descriptor and the dependencies the output of the processor algorithm is the graph model that represent the data of the form. The task of this graph model is to generate the data coming from the form submission, and make it available for client by the editing. The set of triples of the descriptor dataset builds a graph that can be divides into subgraph by the multi triples. The \textit{Graph} is class in the framework that incorporates the list of triples belongs to it, the as well the subgraph objects. This means in the first line is the descriptor dataset is converted into an other type of dataset in the server. This graph object has not only the triples it contains further fields that based on the data creation algrorithm can operate.

\img{\V}{graphModel.png}{Graph dataset from triples}{graphModel}


\img{\V}{graphProcess.png}{Main fields of the graph}{graphProcess}


\newpage
\section{Implementation} \label{52}


The description of the implementation starts with the client side because it is more independent from the other parts, and it sufficient to understand properly the functionality of the server. While in the last subsection it is discussed how can the developed framework integrated into the VIVO web application. 


\subsection{Client side}

This subsection presents the basics of the JavaScript implementation that realizes the dynamic form generation and event handling based on JSON form configuration data. The first part covers the creation of a form itself, and how the data is set to form data object, while the second part is about how the form enables multi dimensional data input by means of sub form adders.
 


\subsubsection{Form loading}


In contrast to Java, JavaScript codes are not necessarily built up in an object-oriented manner. On pages where the elements are statically defined in HTML, it is sufficient to assign event handler routines to them. However in our case none of the elements of the page is coded into HTML, but everything is dynamic and thus added by JavaScript. In the implementation JavaScript classes are applied, whose input is the descriptor object, based on which they generate the corresponding data input fields, and handles the data entered through them by the user. In this section the functionality of the two main classes, the \textit{Form} and \textit{Formelement} is discussed.  

\img{\V}{JS_form_formElement.png}{Form and form element classes}{JS_form_element}

\figref{JS_form_element} illustrate the structure of the two main classes. The most fundamental difference betwen these JavaScript classes to Java classes, that they do not contain only fields and routines (or methods) but UI elements as well. The UI elements can represent and HTML tag, and can be added or removed any time by the routines. Each class of the implemented JavaScript library has a defined set of UI elements. 

The form generation processes starts with the initialization of a \textit{Form} object, where the constructor (like in Java) gets the descriptor JSON object coming from the server. As it was described in the previous chapter the descriptor contains a list of the form element descriptor objects. The \textit{loadFormElements()} routine iterates through on this array, and initiates the form element objects.

\begin{lstlisting}[basicstyle=\footnotesize, frame=single, caption={Form generation based on configuration data}, label=formGen captionpos=b, belowskip=1em, aboveskip=2em]
var formData = new Object()
for(var i = 0, i < formElements.length; i++){
	switch(formElements[i].type){
		case "stringField":
			var element = new StringField(formElements[i], formData)
			break;
		case "selector" :  ...
	}
	this.elements.append(element.container)
}	
\end{lstlisting}


Each form element type is represented as subclass of the \textit{formElement} class. They all have a container UI field, that contains their title and input field HTML element. This container field is added to the \textit{elements} field of the \textit{Form} object.

Listing \ref{formElement} show a small cut from the code of the StringField, which is the subclass of the \textit{FormElement} class. The field \textit{inputField} is the HTML <input/> tag, and if its value changes then the \textit{editHandler} routine is called. The editHandler is the function that realizes the dynamic form data creation, by setting the value of the input field into the form data object with the key defined in its the descriptor. The key is stored in the \textit{dataKey} field of the descriptor, which is the variable name of the RDFNode the input element represents.


\begin{lstlisting}[basicstyle=\footnotesize, frame=single, caption={Form element}, label=formElement, captionpos=b, belowskip=1em, aboveskip=2em]
class StringField extends FormElement{
	constructor(descriptor, formData){
		super(descriptor, formData)
		this.inputField = $("<input/>").type("text").change(this.editHandler)
		...
	}
	editHandler(){
		this.formData[this.descriptor.dataKey]=this.inputField.val() 
	}			
}
\end{lstlisting}

This is the basic mechanism of how object-oriented JavaScript can be employed to generate forms, and put the entered values in to JSON object based on configuration data.


\subsubsection{Sub forms}


The previous section explained how the form algorithm creates the JSON object of the form data. This section extends the explanation of how it is possible to add the multi level data by sub form adders. To this two new JavaScript class functionality is outline, the \textit{SubFormAdder} and the \textit{SubForm}. The former is the subclass of the \textit{FormElement} and the latter of the \textit{Form} class. 

\img{\V}{JS_subFormAdder.png}{SubForm and sub form adder}{JS_subFormAdder}


\figref{JS_subformAdder} depicts the UI elements of the two classes. The routines and fields are inherited from the parent classes.


\begin{lstlisting}[basicstyle=\footnotesize, frame=single, caption={Sub form adder routine}, captionpos=b, belowskip=1em, aboveskip=2em]
class SubformAdder {
	
	constructor(descriptor, formData){
		this.addButton = $("<div/>").text("Add").click(this.add)    
		this.subFormDescriptor = this.descriptor.subForm
		this.formData[this.descriptor.dataKey] = []
	}

	add(){
		var subformDataObject = new Object()
		this.formData[this.descriptor.dataKey].push(subFormDataObject) 
		this.subFormContainer.append(
				new SubForm(this.subFormDescriptor, subFormDataObject))
	}
}
\end{lstlisting}


\begin{itemize}
	\item Important for each subform a new JSON object is generated which is pushed to the array		
%	\item The same way the subform adder checks if the this.formData[this.descriptor.dataKey] contains already existing data and adds them if they are there
\end{itemize}

\subsection{Server side} 

\begin{itemize}
	\item This section introduces how application on the server is able to operate based on data queried from the configuration dataset
	
\end{itemize}

\subsubsection{Overview}

\begin{itemize}
	\item The following image shows the most important steps of the data input process
\end{itemize}

\imgK{\IVII}{overview.png}{Overview of the mechanism on the server}


\subsubsection{Form representation in Java}

\imgK{\IVII}{uml1.png}{UML Diagram of the classes for the form}

\begin{itemize}
	\item The literal fields asks for the type of the variable it represents and the type of the descriptor will be based on this literal field.
\end{itemize}


\imgK{\IVII}{formDescriptor.png}{Form descriptor JSON object}

\subsubsection{Data model in Java}


\begin{itemize}
	\item Querying the configuration triples regarding statements
	\item Processing into the tree structured graph model
\end{itemize}


\imgK{\IVII}{graphUML.png}{Graph UML}

Validation:

\begin{itemize}
	\item Data dependencies can be over graphs
	\item But graphs can be connected only through multi statements
	\item The graph has to be a tree. There are no use cases right now where any loop would be required
\end{itemize}

\imgK{\IVII}{jsonVSGraph.png}{JSON vs graph model}

\begin{itemize}
	\item Data saving mechanism explanation
	\item Data retrieval mechanism explanation
\end{itemize}

\subsubsection{Data Dependencies}


\begin{itemize}
	\item The data dependencies are important only for the form. 
	\item Everything starts with the form descriptor.
	\item There are cases where from the main form no element appears on the form.
	\item A have to bring examples to some problems that illustrate the problem.
	\item Here comes at first the ontology awareness into question....
\end{itemize}


\imgK{\IVII}{exampleProblem.png}{Example problem}

where,

\imgK{\IVII}{explanation.png}{Elements of the simplified notation}

\imgK{\IVII}{restrictionDirection.png}{Two restriction directions}

\begin{itemize}
	\item Post processing for different restriction types
\end{itemize}


\imgK{\IVII}{variableDependency.png}{Getting dependent data}

\subsubsection{Editing and deleting data}

\begin{itemize}
	\item This feature was not introduced on the overview image but it is really important.
\end{itemize}


\imgK{\IVII}{edit.png}{Difference between the submissions and edit data on the form}


\section{VIVO integration} \label{523}

%\img{\IVII}{vivoIntegration.png}{Example problem}{vivoIntegration}
